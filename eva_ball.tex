
\subsection{False-Positives bei der Ballerkennung}
\label{sec:false-positives-bei}
Wie bereits im Abschnitt \ref{sec:balldetection} erwähnt, kann es bei
der Ballerkennung zu False-Positives kommen. Diese liegen in der
Funktionsweise der Ballerkennung begründet: Nach Aufnahme des
Kamerabildes wird das Bild vom RGB- in den HSV-Farbraum umgewandelt,
der anschließend nach roten Partikeln gefiltert wird. Erreichen diese
einen gewissen Schwellwert, werden diese nach kreisförmigen Objekten
durchsucht. Anschließend wird die Wahrscheinlichkeit berechnet, ob es
sich dabei um einen Ball handelt. Wenn nun aber im Kamerabild so ein
rotes Objekt ist (z.B: Ein anderer p3dx-Roboter), wird dieses mit
hoher Wahrscheinlichkeit als ,,Ball'' erkannt. In unseren Versuchen
wurden unter anderen der Feuerlöscher des Labors, aber auch die roten
Schränke gerne als ,,Bälle'' erkannt. Dies konnten wir durch Aufbau
eines geeigneten Sichtschutzes verhindern, allerdings kam es immer
noch zu False-Positives an Stellen, wo weder ein rotes Objekt, noch
sonst irgendein Objekt vorhanden war. Wir konnten schließlich durch
Rückfrage mit Tobias Breuer klären, dass dann höchstwahrscheinlich im
Licht der den Vorraum des Fahrstuhl vor dem Robotiklabor beleuchtenden
Lampen soviele Rot-Partikel vorhanden sind, dass auch da unter
Umständen False-Positives auftauchen können. Aus Zeitgründen war es
uns leider nicht mehr möglich zu testen, ob das Ausschalten der Lampen
zu einer Verbesserung geführt hätte. An und für sich wäre das aber die
logische Konsequenz aus der vermuteten Ursache und auch die leichteste
Möglichkeit, diese Vermutung auf ihren Wahrheitsgehalt zu überprüfen.

%%% Local Variables:
%%% mode: latex
%%% TeX-master: "template"
%%% End:
