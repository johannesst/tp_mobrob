\chapter{Integration bestehender Projekte in ein neues Buildsystem für den
Client}

\section{Warum ein neues Buildsystem?}
Die Ballerkennung und der Sonarpartikelfilter hängen von vielen Altprojekten
ab. Das Beispiel für die Steuerung der Pioneer 3-DX hing zudem von der ARIA
Bibliothek\footnote{http://robots.mobilerobots.com/wiki/ARIA} ab. Bei den
Altprojekten fanden sich auch zum Teil fest kodierte Pfade. Da auch die
Konvertierung der Solutions in das VS 2010 Format meistens fehlschlug und auch
ARIA nicht mit der mitgelieferten Solution unter VS 2010 gebaut werden konnte,
haben wir uns für den Client für den Einsatz eines neuen Buildsystems
entschlossen. Aufgrund der Einfacheit der Beschreibung der Buildvorgänge für
die einzelnen zu integrierenden Projekte und dem Hinzufügen neuer
Unterprojekte, sowie der Möglichkeit, von einem konkreten Buildsystem abhängig
zu sein sowie existierender Erfahrungen mit dem System, fiel die Wahl auf
CMake\footnote{http://www.cmake.org/}. Bei CMake handelt es sich um ein
open-source Metabuildsystem. Es erzeugt Build-Vorschriften für andere
Buildsysteme (u.a. GNU make, nmake, msbuild). Somit wäre es bei Bedarf auch
Möglich gewesen, auf eine andere Visual Studio Version umzusteigen ohne
die Solutions zu konvertieren oder neu zu erstellen.

\section{Integration der Projekte}
Zuerst wurde eine passende Ordnerstrukur angelegt: include für Header-Dateien,
src für die cpp-Dateien der einzelnen Projekt, bin für eventuell zum Linken benötigte
Kompillate, deren Quellen nicht in den Build-Prozess integrierbar waren und
build als Zielordner für erzeugte Buildfiles.

Als nächstes wurde für das Projekt im Ordner des Clients eine
CMakeLists.txt-Datei angelegt:

\begin{lstlisting}
cmake_minimum_required (VERSION 2.6) 
project (TP_WS_2010) 

include_directories (include)
include_directories (include/ControlExample)
include_directories (include/erob/)
include_directories (include/Balldetection)
include_directories (include/p3dxSteuerung)
include_directories (include/StochasticLib)
include_directories (include/PioneerMRClient)
include_directories (include/Aria)
include_directories (include/OccupancyGridMap)
include_directories (include/irpVideo/DirectShow)
link_directories (${TP_WS_2010_BINARY_DIR})
link_directories (${TP_WS_2010_SOURCE_DIR}/lib/)
add_subdirectory (src)
\end{lstlisting}

Diese definiert den Namen des Projektes (wichtig
für Referenzen auf z.b. das Stammverzeichnis des Projektes oder das
Buildverzeichnis, da die entprechenden Variablennamen mit dem Projektnamen + \_
als Präfix versehen werden. Zudem wurden die include-Verzeichnisse und zum
Linken relevante Verzeichnisse festgelegt. Danach wurde das src-Verzeichnis als
Projektunterordner hinzugefügt. Dadurch werden die Anweisungen in der
CMakeLists.txt des Unterordners auch ausgeführt. Dies ermöglicht es, für die
einzelnen Unterprojekte jeweils eine eigene kleine CMakeLists.txt zu verwenden.

Die CMakeLists.txt bindet nun alle Sourceordner der Unterprojekte ein:

\begin{lstlisting}
cmake_minimum_required (VERSION 2.6)

add_subdirectory (irpUtils)
add_subdirectory (irpMath)
add_subdirectory (irpImage)
add_subdirectory (irpVideo)
add_subdirectory (irpFeatureExtraction)
add_subdirectory (alglib)
add_subdirectory (DistanceMap)
add_subdirectory (irpCamera)
add_subdirectory (irpV3d)
add_subdirectory (StochasticLib)
add_subdirectory (Visualization)
add_subdirectory (CamCalib)
add_subdirectory (Balldetection)
add_subdirectory (mathtest)
add_subdirectory (SonarParticleFilter)
add_subdirectory (PioneerMRClient)
add_subdirectory (OccupancyGridMap)
add_subdirectory (Balldetection_Test)
add_subdirectory (Aria)
add_subdirectory (p3dxSteuerung)
add_subdirectory (p3dxSteuerung_threaded)
add_subdirectory (calibrate)
#add_subdirectory (ControlExample)
add_subdirectory (utils)
add_subdirectory (kinect)
add_subdirectory (networkTest)
\end{lstlisting}

Bei den Unterprojekten gibt es zwei Arten: Bibliotheken und Anwendungen.

Die CMakeList.txt einer Bibliothek sieht wie folgt aus:

\begin{lstlisting}
cmake_minimum_required (VERSION 2.6) 

file(GLOB src "*.cpp")
file(GLOB includes "../../include/irpCamera/*.h")

add_library (irpCamera ${src} ${includes})
\end{lstlisting}

Die CMakeList.txt einer Anwendung sieht so aus:
\begin{lstlisting}
cmake_minimum_required (VERSION 2.6) 

file(GLOB src "*.cpp")
file(GLOB includes "../../include/p3dxSteuerung/*.h")
add_executable (p3dxSteuerung_threaded ${src} ${includes})
target_link_libraries (p3dxSteuerung_threaded irpUtils MINPACK fftw rfftw  DirectShow irpFeatureExtraction irpMath  alglib irpImage irpCamera irpVideo irpV3d glew CamCalib Visualization StochasticLib OccupancyGridMap SonarParticleFilter ws2_32 winmm advapi32 Aria DistanceMap Balldetection)

\end{lstlisting}

Zuerst werden alle cpp-Dateien zur Variablen src hinzugfügt, danach die
Includes zur Variablen include. Je nachdem, ob es sich um eine Bibliothek oder
Anwendung handelt, wird diese mit add\_library(name dateien) oder
add\_executable(name dateien) hinzugefügt. Die Include-Dateien müssen dabei
aber eigentlich nicht explizit hinzugefügt werden, dies geschieht nur, damit
sie auch in einer VS Solution in der Dateiliste auftauchen. Bei Anwendungen
können nun mit target\_link\_libraries(exename libraries) noch die zu linkenden
Bibliotheken angegeben werden.

Es gab ein paar Probleme bei der Integration: zum einen mussten zuerst die
Abhängigkeiten zwischen den Unterprojekten ermittelt werden, dies geschah mit
Hilfe der originalen VS Solutions sowie durch Ausprobieren. Zum anderen gab es
manchmal Dateien, die zwar in den Quellen lagen, jedoch in nicht in den
Solutions auftauchten. Diese konnten den Buildprozess stören und mussten daher
entfernt werden TODO: Beispiel raussuchen, wenn noch überhaupt möglich.
Außerdem waren an ein paar Stellen kleinere Änderungen am Quellcode nötig, um
ihn mit MSVC 2010 oder der Struktur unseres include-Ordners kompatibel zu machen.

\section{HOWTO: Hinzufügen eines neuen Unterprojektes}
Das Hinzufügen eines neuen Unterprojektes ist relativ einfach.
Zuerst wird ein Unterverzeichnis für das Projekt im include-Ordner und eines im
src-Ordner angelegt. Dann kopiert man die includes in den neuen Unterordner im
include-Ordner. Dabei ist darauf zu achten, dass die Pfade in
\#include-Direktiven relative zum include-Ordner sein sollten. Danach kopiert
man die cpp-Dateien in den neuen Unterordner im src-Verzeichnis.

Danach erstellt man in diesem Ordner eine CMakeLists.txt ähnlich wie oben,
jenachdem ob ex sich um eine Anwendung oder Bibliothek handelt. Man fügt zuerst
die cpp-Dateien und Includes hinzu, fügt dann mit add\_library oder
add\_executable Buildtargets hinzu und gibt eventuell noch zu linkende
Bibliotheken bei ANwendungen an mit target\_link\_libraries.

\section{HOWTO: Bauen des Clients mit VS 2010}
TODO: Bauen mit der CMake GUI beschreiben
