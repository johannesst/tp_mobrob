
\chapter{Softwarearchitektur}
\label{cha:softwarearchitektur}

\section{Allgemeiner Aufbau}
\label{sec:allgemeiner-aufbau}

Die Software besteht aus zwei Komponenten:
\begin{itemize}
\item Der Server, der die Bahnplanung für die Roboter, sowie die
  Visualisierung der Ballsuche übernimmt.
\item Der Client, der die Ballerkennung, sowie Positionsbestimmung
  vornimmt. Ausserdem ist er dafür verantwortlich, den Roboter zu
  steuern. Dazu lässt er den Roboter die durch den Server vorgegebenen
  Punkte der geplanten Bahn anfahren. 
\end{itemize}
b
\section{Server}
\label{sec:server}

\section{Client}
\label{sec:client}

\subsection{SonarParticleFilter}
\label{sec:sonarparticlefilter}
Hierbei handelt es sich um ein vorheriges Teamprojekt, auf dass wir
zurückgreifen konnten. Es nutzt das im Roboter integrierte
Sonarsystem, um anhand einer Karte die Position des Roboters im Raum
zu bestimmen. Dazu bedienen wir uns seiner Klasse particleSet. 
Mit dem Konstruktor \lstinline|particleSet::particleSet(const char
*map_path, const char *wskTxT_path, bool use_ray_casting, int max_x,
int min_x, int max_y, int min_y,int map_size, int map_res, int
numofParticle, ArRobot *p3dx)| erzeugen wir ein Objekt
\lstinline|particleSet| mit der Partikelmenge. Dabei übergeben wir ihm auch den Pfad zur
Karte des Raumes und legen fest, ob wir ray_casting benutzen oder
nicht, sowie die Auflösung und Koordinaten der Karte.  \\
Nach Erzeugen des Partikelmengen-Objektes können wir dann mit der
Methode \lstinline|particleSet::startFilter()| die aktuellen
Koordinaten des Roboters abfragen.
\subsection{Balldetection}
\label{sec:balldetection}



%%% Local Variables: 
%%% mode: latex
%%% TeX-master: "template"
%%% End: 
