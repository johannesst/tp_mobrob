\subsection{Ballerkennung}

\label{sec:balldetection}
Die Ballerkennung läuft in einen eigenen Thread im Client, da sie und
die Fahrschleife des Roboters sich gegenseitig blockieren
würden. Dabei greift sie auf das Kamerabild einer Logitech Webcam
(TODO Typ einfügen!) zurück. Nach Initialisierung der Ballerkennung
werden folgende Schritte durchgeführt, bis der Ball gefunden oder die
Suche abgebrochen wurde:
\begin{enumerate}
\item Aktualisieren des aktuellen Kamerabildes
\item Aufruf der Erkennungsmethode der Balldetection
  \lstinline|detectBall()|. 
\item Abfrage, ob der Ball erkannt wurde mit der Methode
  \lstinline|getActualPicProb()|. Liefert diese eine
  Wahrscheinlichkeit zurück, die größer als 80 \% ist, ist die Suche
  beendet und die Koordinaten des Balls werden wie folgt bestimmt:
  \begin{enumerate}
  \item Mit den Methode \lstinline|getDistanceToBall()| und
    \lstinline|getAngleToBall()| wird die Entfernung und der Winkel
    zum Ball abgerufen. Daraus und der aktuellen Position des Roboters
    wird die Ballposition bestimmt.
  \item Danach wird die gefundene Position zum Server übertragen
  \end{enumerate}
\item Ist die Wahrscheinlichkeit hingegen kleiner als 80 \% wird mit
  der Ballerkennung fortgefahren. 
\end{enumerate}

Wie bereits erwähnt greifen wir bei der Ballerkennung auf die
Bachelorarbeit von Tobias Breuer zurück. Dadurch konnten wir einfach
die API der von ihm entwickelten Software nutzen, ohne uns selber mit
den Details der Ballerkennung beschäftigen zu müssen. Sie verfolgt
folgenden Ansatz: Im Kamerabild wird nach roten Pigmenten
gesucht. Erreichen diese eine gewisse Anzahl, sucht die Software nach
kreisförmigen Objekten, die diese Pigmente enthalten. Je nachdem,
wieweit diese Kriterien erfüllt sind, berechnet die Software eine
Wahrscheinlichkeit, ob sich im aktuellen Bild ein Ball befindet und
die Entfernung und den Winkel zum vermuteten Ball. Um vernünftige
Werte zu erhalten ist es unabdingbar, die Software vorher zu
kalibrieren. Nichts destotrotz kann es zu false Positives
kommen. Näheres zur Kalibrierung sowie zur Häufigkeit und Ursachen von
false Positives finden sich im Abschnitt \ref{cha:evaluation} ab Seite \pageref{cha:evaluation}.
%%% Local Variables: 
%%% mode: latex
%%% TeX-master: "template"
%%% End: 
