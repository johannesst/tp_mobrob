\chapter{Der Client}
Der Client in unserer Client-Server-Architektur ist für das Anfahren der
einzelnen Routenpunkte und die Erkennung des Balls zuständig. Dazu haben
wir für die Lokalisierung die uns zu Verfügung gestellte Implementierung
eines Sonar-Partikelfilters und für die Ballerkennung die Implementierung
aus einer Bachelorarbeit von Tobias Breuer verwendet.

Am Partikelfilter haben wir ein paar Änderungen vorgenommen. Es gab keine
Möglichkeit, von außerhalb des Filters auf die Koordinaten und die
Ausrichtung des Roboters zuzugreifen. Dazu wurden die Methoden StartFilter
und findBestParticle so erweitert, dass die aktuelle Position des Roboters
und seine Ausrichtung als Zeiger auf das erste Element eines double-Arrays
([x,y,Theta, Partikelwsk.]) zurückgegeben wird. Da dieses in
findBestParticle dynamisch mit new alloziiert wird, muss es, wenn es nicht
mehr beötigt wird, mit delete[] gelöscht werden. Außerdem ist nun möglich
die Anzahl der Sonare von außerhalb des Filters ohne neukompillieren
einzustellen. Die Visualisierungen wurden aus Performancegründen entweder
entfernt oder deaktiviert. Zudem wurde für eine einfachere Auswertung und
weil die Partikelvisualisierung öfter zu Abstürzen unseres Programms
geführt hat der Filter um eine Funktion erweitert, die, wenn in der
particleSet.cpp takeADump true ist, bei jedem Filtern alle Partikel als
Tripel aus x,y und der Partikelwsk. in eine Datei schreibt. Die Datei hat
dabei als Prefix eine Zahl im Namen, die bei 0 beginnend nach jedem Filtern
 um 1 erhöht wird. Passend dazu wurde ein Visualisierer geschrieben, der
die Partikelmenge grafisch darstellt, auf den noch später eingegangen wird.

