\subsection{Lokalisierung mit den Sonar-Partikelfilter}
\label{sec:lokal-mit-den}
Bei unseren Versuchen die Position des Roboters im Raum zu bestimmen,
stiessen wir recht früh auf Probleme: So wurde regelmäßig eine
komplett falsche Position im Raum bestimmt oder aber (gerade wenn der
Roboter sich nahe einer Wand befand) die dem Roboter gegenüberliegende
Position am anderen Ende des Raums als Position erkannt.  Nach
mehreren Versuchen stellten wir schließlich fest, dass die
Lokalisierung besonders zuverlässig war, wenn der Roboter beim Start
des Clients sich in der Mitte des Raums befand, sodass bei der
Initialisierung des Partikelfilters durch die initiale Rotation der
Abstand zu den Wänden zu beiden Seiten gleich war. Um nun der Ursache
dieses Phänomens auf die Spur zu kommen, haben wir dann unser eigenes
Visualisierungsskript geschrieben. Wir beschreiben nun zunächst die
Funktionsweise und Installation des Skriptes, bevor wir uns den
Ergebnissen zuwenden.
\section{Visualisierung der Partikelmengen des Partikelfilters}
Da die im Partikelfilter integrierte Visualisierung der Partikelmenge
nicht immer funktionierte und es auch keine direkte SPeichermöglichkeit der
 Bilder gab, haben wir den Partikelfilter, so erweitert, dass die
Partikelmenge in eine Datei geschrieben werden kann (
\ref{sec:sonarparticlefilter}) und passend dazu ein Skript (in Python unter
 Benutzung von pygame geschrieben), welches die Partikel anhand der Daten
 aus einer Partikelmengendatei über ein Bild legt. Dieses ist im Ordner
 Visualisierung im SVN-Repository unter dem Namen visualize.py zu finden.

 \subsection{HOWTO: Installieren der Abhängigkeiten des
Visualisierungsskriptes und Benutzung dessen}
 \begin{itemize}
	 \item Python 3.2 herunterladen und installieren\footnote{http://python.org/ftp/python/3.2.2/python-3.2.2.msi}
	 \item pygame 1.9.2a0 für Python 3.2 installieren \footnote{http://pygame.org/ftp/pygame-1.9.2a0.win32-py3.2.msi}
	 \item Python 3.2 zum PATH hinzufügen
	 \item cmd/Eingabeaufforderung öffnen
	 \item In den Ordner Visualisierung des Repositories wechseln
	 \item Das Visualisierungsskript kann nun folgendermaßen benutzt werden:\\
	 		\lstinline|python visualize.py (Dateiname des Bildes) (Dateiname der Partikelmengendatei)| \\
 			z.b. \lstinline|python visualize.py OccuMap.bmp visual.log|
	 \item Das Ausgabebild hat dann den Namen V\_(Dateiname der Partikelmengendatei)\_(Dateiname des Bildes)
\end{itemize}

%\section{Visualisierung der Partikelmengen des Partikelfilters}
Da die im Partikelfilter integrierte Visualisierung der Partikelmenge
nicht immer funktionierte und es auch keine direkte SPeichermöglichkeit der
 Bilder gab, haben wir den Partikelfilter, so erweitert, dass die
Partikelmenge in eine Datei geschrieben werden kann (
\ref{sec:sonarparticlefilter}) und passend dazu ein Skript (in Python unter
 Benutzung von pygame geschrieben), welches die Partikel anhand der Daten
 aus einer Partikelmengendatei über ein Bild legt. Dieses ist im Ordner
 Visualisierung im SVN-Repository unter dem Namen visualize.py zu finden.

 \subsection{HOWTO: Installieren der Abhängigkeiten des
Visualisierungsskriptes und Benutzung dessen}
 \begin{itemize}
	 \item Python 3.2 herunterladen und installieren\footnote{http://python.org/ftp/python/3.2.2/python-3.2.2.msi}
	 \item pygame 1.9.2a0 für Python 3.2 installieren \footnote{http://pygame.org/ftp/pygame-1.9.2a0.win32-py3.2.msi}
	 \item Python 3.2 zum PATH hinzufügen
	 \item cmd/Eingabeaufforderung öffnen
	 \item In den Ordner Visualisierung des Repositories wechseln
	 \item Das Visualisierungsskript kann nun folgendermaßen benutzt werden:\\
	 		\lstinline|python visualize.py (Dateiname des Bildes) (Dateiname der Partikelmengendatei)| \\
 			z.b. \lstinline|python visualize.py OccuMap.bmp visual.log|
	 \item Das Ausgabebild hat dann den Namen V\_(Dateiname der Partikelmengendatei)\_(Dateiname des Bildes)
\end{itemize}

%%% Local Variables:
%%% mode: latex
%%% TeX-master: "template"
%%% End:
