\subsection{Sonar-Partikelfilter}
\label{sec:sonarparticlefilter}
Es nutzt das im Roboter integrierte
Sonarsystem, um anhand einer Karte die Position des Roboters im Raum
zu bestimmen. Dazu bedienen wir uns seiner Klasse particleSet. 
Mit dem Konstruktor \lstinline|particleSet::particleSet(const char *map_path, const char *wskTxT_path, bool use_ray_casting, int max_x, int min_x, int max_y, int min_y,int map_size, int map_res, int numofParticle, ArRobot *p3dx)| erzeugen wir ein Objekt
\lstinline|particleSet| mit der Partikelmenge. Dabei übergeben wir ihm auch den Pfad zur
Karte des Raumes und legen fest, ob wir Raycasting benutzen oder
nicht, sowie die Auflösung und Koordinaten der Karte.  \\
Nach Erzeugen des Partikelmengen-Objektes können wir dann mit der
Methode \lstinline|particleSet::startFilter()| die aktuellen
Koordinaten des Roboters abfragen. \\\\
Am Partikelfilter haben wir ein paar Änderungen vorgenommen. Es gab keine
Möglichkeit, von außerhalb des Filters auf die Koordinaten und die
Ausrichtung des Roboters zuzugreifen. Dazu wurden die Methoden
\lstinline|double* StartFilter()|
und \lstinline|double* findBestParticle()| so erweitert, dass die aktuelle Position des Roboters
und seine Ausrichtung als Zeiger auf das erste Element eines Arrays
\lstinline|double* ([x,y,Theta, Partikelwsk.])| zurückgegeben wird. Da dieses in
\lstinline|double * findBestParticle()| dynamisch mit new alloziiert wird, muss es, wenn es nicht
mehr beötigt wird, mit \lstinline|delete[]| gelöscht werden. Außerdem ist nun möglich
die Anzahl der Sonare von außerhalb des Filters ohne neukompillieren
einzustellen. Die Visualisierungen wurden aus Performancegründen entweder
entfernt oder deaktiviert. Zudem wurde für eine einfachere Auswertung und
weil die Partikelvisualisierung öfter zu Abstürzen unseres Programms
geführt hat der Filter um eine Funktion erweitert, die, sofern dies
beim Kompilieren der Software aktiviert wurde,  bei jedem Filtern alle Partikel als
Tripel aus x,y und der Partikelwsk. in eine Datei schreibt. Die Datei hat
dabei als Prefix eine Zahl im Namen, die bei 0 beginnend nach jedem Filtern
 um 1 erhöht wird. Passend dazu wurde ein Visualisierer geschrieben, der
die Partikelmenge grafisch darstellt, auf den noch später eingegangen
wird.

%%% Local Variables: 
%%% mode: latex
%%% TeX-master: "template"
%%% End: 
