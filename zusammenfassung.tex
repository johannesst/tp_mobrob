
\chapter{Zusammenfassung}
\label{cha:zusammenfassung}
Unsere Aufgabenstellung eine koordinierte Ballsuche 
zu implementieren, ist uns im Wesentlichen gelungen:\\
Ein zentraler Server nimmt die Bahnplanung vor und ist in der Lage die
Bahnplanung für bis zu drei Clients vorzunehmen. Diese Bahn wird von
der Clientsoftware ohne größere Probleme abgefahren. Mit der Hilfe der
uns zur Verfügung stehenden Ballerkennung von Tobias Breuer und des
auf einen früheren Teamprojekt aufbauenden Sonar-Partikelfilters war
es möglich, den Client so zu
entwickeln, dass er nicht nur seine eigene Position im Raum, sondern
auch eventuell dort vorhandene Bälle bestimmen kann. \\\\
Zwar erwies sich der Partikelfilter in bestimmten Situationen als nur
bedingt zuverlässig, diese Problematik lässt sich aber durch ein
geschickte Wahl der Initialisierungsposition umgehen. Theoretisch wäre
es auch denkbar, die Roboter zusätzlich zum eingebauten Sonar mit
einen Laserscanner auszustatten, dieser sollte vom Ansatz her deutlich
zuverlässigere Ergebnisse liefern. Eine Implementierung dieses
Ansatzes lag allerdings nicht im Fokus unseres Projektes. \\\\
Grundsätzlich kann also das Projekt als Erfolg betrachtet werde.n
%%% Local Variables:
%%% mode: latex
%%% TeX-master: "template"
%%% End:
