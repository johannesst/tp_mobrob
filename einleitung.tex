
    \chapter{Einleitung}
    \label{einleitung}
    \pagenumbering{arabic}
    Unsere Aufgabe ist es, eine koordinierte Ballsuche in einen Raum
    zu implementieren. \\\\

Das Ziel der koordinierten Suche besteht darin, dass die im Raum bzw. im
Suchterrain befindlichen mobilen Roboter miteinander das gesuchte Ziel
finden. Diese Suche setzt voraus, dass sich die Roboter kollisionsfrei
im Raum bewegen. Sie dürfen weder mit Objekten aus der Umgebung, noch
mit anderen Robotern, die ebenfalls auf der Suche nach dem gegebenen
Objekt sind, kollidieren. Zur Umsetzung dieser Aufgabe stanen eine
Ad-hoc-Lösung und schließlich eine Server-Client-Lösung zu Auswahl, auf
die die Entscheidung fiel. Dies ermöglicht eine paaralele Entwicklung
von Client und Server, was eine einfachere Aufgabenverteilung im Team
ermöglicht. Ausserdem kann dann auf den Server eine Visualisierung der
Ballsuche erfolgen, indem eine Karte des Raumes, sowie der darin
befindlichen Roboter sowie (nach erfolgreicher Suche) des Balls in der
GUI des Servers angezeigt wird. Der Client ist dann nur noch für das
Abfahren der von Server vorgebenen Postionen, sowie die Lokalisierung
im Raum und Ballerkennung in seiner unmittelbaren Umgebung
zuständig. \\\\
Zum Erreichen des Ziels standen uns bis zu drei mobile Roboter 
    ,,Pioneer-3DX'' der Firma ,,adept mobilerobots'' zur
    Verfügung. Dazu konnten wir diverse Libaries des Instituts
    nutzen. Die wichtigsten schon vorhandenen Komponenten waren aber
    der SonarPartikelfilter, sowie die Ballerkennung:
    \begin{itemize}
    \item Der SonarPartikelfilter ging auf ein vorheriges Teamprojekt
      zurück. Mit ihm konnten wir die Position des Roboters im Raum
      relativ  genau bestimmen. Diese diente zum einen als
      Basis für die Bahnplanung. Zum anderen war dies die Basis zur
      Bestimmung der Position eines erkannten Balls.
    \item Die Ballerkennung konnten wir aus der Bachelorarbeit von
      Tobias Breuer übernehmen. Sie stelle uns eine API zur Verfügung,
      worüber wir den Ball finden, sowie seine genaue Position
      bestimmen konnten.
    \end{itemize}
Unsere Softwarearchitektur bestand somit aus folgenden Komponenten:
\begin{itemize}
\item Der Server: Er nimmt die Bahnplanung, sowie Kollisionsvermeidung
  vor und steuert bis zu drei Clients. Gleichzeitig dient er als
  Visualisierung des aktuellen Status der Ballsuche.
\item Der Client: Eine Steuersoftware für die Roboter. Sie ist dafür
  zuständig, die vom Server vorgebene Bahnplanung umzusetzen. Dafür
  nutzt er den SonarPartikelfilter, um seine eigene Position im Raum
  zu bestimmen. Außerdem nimmt er die eigentliche Ballerkennung vor
  und bestimmt aus den dabei erhalten Informationen und der durch den
  Partikelfilter erhaltenen Position die Position des Balles im Raum.es
\end{itemize}
Im nächsten Abschnitt folgt nun eine genauere Beschreibung unserer Software-Architektur
%Es folgt nun die Beschreibung
%%% Local Variables: 
%%% mode: latex
%%% TeX-master: "template"
%%% End: 
