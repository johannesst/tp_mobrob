
    \chapter{Einleitung}
    \label{einleitung}
    \pagenumbering{arabic}
    Unsere Aufgabe ist es, eine koordinierte Ballsuche in einem abgegrenzten Raum
    zu implementieren. \\\\

Das Ziel der koordinierten Suche besteht darin, dass die im Raum bzw. im
Suchterrain befindlichen mobilen Roboter miteinander das gesuchte Ziel, einen roten Ball,
finden sollen. Diese Suche setzt voraus, dass sich die Roboter kollisionsfrei
im Raum bewegen. Sie dürfen weder mit Objekten aus der Umgebung, noch
mit anderen Robotern, die ebenfalls auf der Suche nach dem gegebenen
Objekt sind, kollidieren. Zur Umsetzung dieser Aufgabe standen eine
Ad-hoc-Lösung und schließlich eine Server-Client-Lösung zu Auswahl, auf
die die Entscheidung fiel. Dies ermöglicht eine parallele Entwicklung
von Client und Server, was eine einfachere Aufgabenverteilung im Team
ermöglicht. Ausserdem kann dann auf dem Server eine Visualisierung der
Ballsuche erfolgen, in dem eine Karte des Raumes, sowie die Position des darin
befindlichen Roboter sowie (nach erfolgreicher Suche) die Position des Balls in der
GUI des Servers angezeigt wird. Der Client ist dann nur noch für das
Abfahren der von Server vorgebenen Postionen, sowie die Lokalisierung
im Raum und Ballerkennung in seiner unmittelbaren Umgebung
zuständig. \\\\
Zum Erreichen des Ziels standen uns mobile Roboter 
    ,,Pioneer-3DX'' der Firma ,,adept mobilerobots'' zur
    Verfügung. Dazu konnten wir diverse Bibliotheken des Instituts
    nutzen. Die wichtigsten schon vorhandenen Komponenten waren aber
    der SonarPartikelfilter, sowie die Ballerkennung:
    \begin{itemize}
    \item Der SonarPartikelfilter geht auf ein vorheriges Teamprojekt
      zurück. Mit ihm konnten wir die Position des Roboters im Raum
      relativ  genau bestimmen. Sie dient als Basis für die Bahnplanung
      und zum Anfahren der durch diese bestimmten Zielpunkte.
      %sie als Basis für seine Bahnplanung nutzt
      %Basis für die Bahnplanung.
    \item Die Ballerkennung konnten wir aus der Bachelorarbeit von
      Tobias Breuer übernehmen. Sie stellte uns eine API zur Verfügung,
      worüber wir den Ball finden, sowie seine genaue Position
      bestimmen konnten.
    \end{itemize}
Unsere Softwarearchitektur bestand somit aus folgenden Komponenten:
\begin{itemize}
\item Der Server: Er nimmt die Bahnplanung, sowie Kollisionsvermeidung
  vor und steuert bis zu drei Clients. Gleichzeitig dient er als
  Visualisierung des aktuellen Status der Ballsuche.
\item Der Client: Eine Steuersoftware für die Roboter. Sie ist dafür
  zuständig, die vom Server vorgebene Bahnplanung umzusetzen. Dafür
  nutzt er den Partikelfilter für Ultraschallmessungenr, um seine eigene Position im Raum
  zu bestimmen. Außerdem nimmt er die Ballerkennung vor
  und bestimmt aus den dabei erhalten Informationen und der durch den
  Partikelfilter erhaltenen Position die Position des Balles im Raumes.
\end{itemize}
Im nächsten Abschnitt \ref{cha:softwarearchitektur} folgt nun eine genauere Beschreibung unserer
Software-Architektur.
%Es folgt nun die Beschreibung
%%% Local Variables: 
%%% mode: latex
%%% TeX-master: "template"
%%% End: 
