\documentclass[a4paper]{report}
\usepackage{a4}
\usepackage{doku}

\usepackage{longtable}
	\usepackage[ngerman]{babel}
	\usepackage[numbers,sort&compress]{natbib}
%\usepackage[draft]{graphicx}
\usepackage{nofloat}
\usepackage{graphicx}
\usepackage{tikz}
%\usepackage{graphics}
\usepackage{color}
\usepackage[dvips]{epsfig}
\usepackage{bbm}
\usepackage[utf8]{inputenc}
\usepackage{nomencl}
\usepackage[normalem]{ulem}
\usepackage[]{listings}
\usepackage{setspace}
\usepackage{nomencl}
\usepackage[normalem]{ulem}
\usepackage{subfigure} 
\usepackage{subfig} 
%	\usepackage{graphics}
\usepackage{color}
\usepackage{amsmath}
\usepackage{amssymb} 
\usepackage{setspace}
\usepackage{pifont} %Für ein Kreuz \ding{55}
\usepackage{amsfonts} %Für ein Haken \checkmark
\usepackage{multirow} 	
\usepackage{algorithmic}
\usepackage{algorithm}
\usepackage{ifpdf}
	\usepackage{placeins}
	\usepackage{bbm}
\newcommand{\ve}[1]{\mbox{\boldmath$#1$}}
\newcommand{\ma}[1]{\mbox{\boldmath$#1$}}
\newcommand{\eR}{\mbox{$\varepsilon I\;R$}}
\renewcommand{\today}{10.Dezember 2011}
	\definecolor{darkgray}{rgb}{0.95,0.95,0.95}
	\definecolor{darkgreen}{rgb}{0.0,0.48,0.0}
	\lstset{language=C++,
		aboveskip = 10pt,
		belowskip = 10pt,
		backgroundcolor=\color{darkgray},
		numbers=left, numberstyle=\small, 
		stepnumber=1, numbersep=8pt,
		columns = fixed,
		keywordstyle=\color{blue},
		commentstyle=\color{darkgreen},
		basicstyle=\ttfamily\small,
		showspaces = false,
		showtabs = false,		
		lineskip=-1pt,
		tabsize=2}

\begin{document}

\begin{titlepage}
        \setcounter{page}{1}
        \let\footnotesize\small
        \let\footnoterule\relax
        \headsep 1.5cm
        \vskip -4cm
        \centerline{\Huge\bf Technische Universit\"at Braunschweig}
        \vskip 3.4cm
        \begin{center}
            \begin{minipage}[t][7cm][c]{13.5cm}
                \begin{center}
                    {\large{Teamprojekt}\par}
                    \vskip 0.5cm
		    {\LARGE\bf Koodinierte Ballsuche mit }
                    \vskip 0.25cm
                    {\LARGE\bf mobilen Robotern}
                    \vskip 0.5cm
                    {\Large\bf Martin Mikolas, Markus Reschke, Johannes
		    Starosta}
                    \vskip 0.25cm


                    {\large\bf Betreuer:
                    \begin{tabular}[t]{c}{René Iser}\end{tabular}}
                    \vskip 0.75cm
                    {\large\bf{\today}\par}
                \end{center}
            \end{minipage}
        \end{center}
        \vskip 1.5cm
        \begin{figure}[h]
            \begin{center}
							\includegraphics{TU-Logo}
            \end{center}
        \end{figure}



        \vskip 1cm
        \centerline{\LARGE\bf Institut f\"ur Robotik und Prozessinformatik}
            \vskip 1cm
        \centerline{\LARGE\bf Prof.~Dr.~F.~Wahl}




    \end{titlepage}

    \declaration{den \today}

    \setcounter{footnote}{0}
    \pagenumbering{roman}
    \begin{abstract}Bei der vorliegenden Ausarbeitung handelt es sich
      um die Dokumentation unseres Teamprojektes zur ,,Koordinierten
      Suche mit mobilen Robotern''. Wir werden zunächst die
      Vorarbeiten vorstellen, auf denen wir ausbauten, bevor wir unser
      System näher vorstellen. Zum Schluss folgt noch eine kritische
      Bewertung unserer Ergebnisse.
    \end{abstract}

%    \raisebox{-12cm}{\hspace{5cm}\huge \textsc{To Loriot}}
 %   \newpage

%    \markright{Danksagungen}
%    \quote{ { \bf \Large  Danksagungen } \originalTeX
%    \\
%    \\
%    Unser Teamprojekt wäre nicht möglich gewesen, wenn wir nicht auf
%    wesentliche Vorarbeiten hätten zirück
%    I am very grateful to many people who helped me during my thesis in various
%    manners.
%    First, I wish to thank the entire staff of the Institute for Robotics and
%    Process Control for their helpful discussions and the truly great working atmosphere.
%    Particularly, I would like to thank Erich Kozlowski for his
%    helpful advice and tremendous encouragement during my student jobs at the Institute
%    and especially throughout this work.
%    Moreover, I am indebted to my family and my friends for their overwhelming
%    support throughout the last years.
%    Furthermore, I wish to thank Hans Meiser and Birte Karalus for their advice on
%    language issues.
%    Finally, I would like to thank the Institut für Mathematische
%    Stochastik, the Institute for Computational Mathematics, and the
%    Institute for Communications Technology for their kind provision
%    of literature.
%    \\ } { Walter Horstmann }


    \tableofcontents
    \listoffigures 
    \lstlistoflistings
    %\listoftables 
    \newpage 
%     \begin{minipage}{16cm} 
%         {\bf \Large Nomenclature} 
%         \\\\\\ 
 
%         {\bf \large Greek Letters\\\\} 
%         \begin{tabular}{ll} 
%                 $\ve{\alpha}$ & Vector denoting the angular acceleration of the sensor frame\\ 
%                 $\alpha_{i}$ & DH-parameter\\ 
%                 $\beta$ & Fudge factor employed in the APF and the DUKF\\ 
%                 $\ve{\chi}$ & Noise sample function\\ 
%                 $\Sigma$& $\Sigma$-points of the UKF to the true noise distribution\\ 
%                 $\delta$ & Fudge factor\\ 
%                 $\Delta$ & Variable denoting a difference\\ 
%                 $\Delta \gamma$ & Angle of rotation around $\ve{\omega}$\\ 
%                 $\Delta t$ & Interval between two sampling points\\ 
%                 $\eta$ & Scaling parameter of the UKF\\ 
%                 $\gamma$ & Influence factor of the Robbins-Monroe update scheme\\ 
%                 $\lambda$ & Forgetting factor in the RLS algorithm; scaling parameter of the DUKF\\ 
%                 $\omega$ & Vector denoting the angular velocity of the sensor frame\\ 
%                 $\omega_{i}$ & Joint angular velocity\\ 
%                 $^{i}\omega_{i}$ & Vector denoting the angular velocity vector of link $i$\\ 
%                 $\ma{\Omega}$ & Matrix parameterizing a sinusoidal 
%                 trajectory\\ 
%                 $\ve{\varphi}$ & Vector containing the inertial 
%                 parameters of a load\\ 
%                 $\ve{\varphi^{dyn}}$ & Inertial parameter vector containing 
%                 all ten inertial parameters\\ 
%                 $\ve{\varphi_{ext}}$ & Inertial parameter vector augmented by 
%                 the force/torque offsets\\ 
%                 $\ve{\varphi^{sta}}$ & Inertial parameter vector containing 
%                 the mass and the products of the mass and the COM coordinates\\ 
%                 $\rho$ & Fudge factor in the $MAD$ calculation\\ 
%                 $\varsigma$ & Factor employed in trajectory 
%                 optimization\\ 
%                 $\sigma$ & Variance of the prediction error; singular value\\ 
%                 $\ve\sigma$ & $\Sigma$-points that predict the 
%                 measurements\\ 
%                 $\ve\sigma$ & $\Sigma$-point that predict the 
%                 measurements\\ 
%                 $\ve\sigma$ & $\Sigma$-point describing the state 
%                 and its prediction\\ 
%                 $\tau$ & Threshold\\ 
%                 $\theta$ & DH-parameter; fudge factor\\ 
%                 $\ve{\upsilon_{k}^{i}}$ & Particle\\ 
%                 $\xi_{i,k}$ & Factor weighting the sine part of 
%                 the sinusoidal joint angle function of joint $i$\\ 
%                 $\zeta_{i,k}$ & Factor weighting the cosine part of 
%                 the sinusoidal joint angle function of joint $i$\\ 
%             \end{tabular} 
%                 \end{minipage} 
% \begin{minipage}{16cm} 
%                 {\bf \Large Nomenclature} 
%                 \\\\\\ 
%                 {\bf \large Roman Letters\\\\} 
%                 \begin{tabular}{ll} 
%                 $\ve{a}$ & Vector denoting the linear acceleration of the sensor frame\\ 
%                 $a_{i}$ & DH-parameter\\ 
%                 $\ma{B}$ & State transition matrix of the Kalman 
%                 filter\\ 
%                 $\ve{c}$ & Coordinates of the center of mass of the load w.r.t. the sensor frame\\ 
%                 $c(P)$ & Constraints function\\ 
%                 $\ma{C}$ & Information matrix\\ 
%                 $d_{i}$ & DH-parameter\\ 
%                 $diag(\ma{X})$ & diagonal matrix with the elements of 
%                 $\ma{X}$ on its main diagonal\\ 
%                 $e$ & Prediction error of the Kalman filter and 
%                 LS-based identification algorithms\\ 
%                 $|\ve{e_{rel}}|$ & absolute value of the relative 
%                 parameter error\\ 
%                 $E[x]$ & Expected value of $x$\\ 
%                 $\ma{E}$ & Error matrix of the identification variables 
%                 according to error model B\\ 
%                 $f$ & frequency\\ 
%                 $f_{exc,i}$ & base frequency of superposed sinusoidal 
%                 functions describing the rotation of joint $i$\\ 
%                 $f_{exc,min}$ & minimum base frequency of the manipulator joints\\ 
%                 $f_{s}$ & Sampling frequency\\ 
%                 $\ve{f}$ & Vector denoting the force exerted by the sensor on the load\\ 
%                 $\ve{f_{o}}$ & Force offset\\ 
%                 $\ma{F_{des}}$ & desired object frame\\ 
%                 $\ma{F_{est}}$ & estimated object frame\\ 
%                 $\ve{g}$ & Gravity vector w.r.t. to the sensor frame\\ 
%                 $g_{0}$ & gravitational constant \\ 
%                 $\ma{I}$ & Identity matrix\\ 
%                 $\ma{J}$ &\begin{math}\ma{J}=\left(\begin{array}{lll}j_{xx} & j_{xy} & j_{xz}\\j_{xy} & j_{yy} & j_{yz}\\j_{xz} & j_{yz} & j_{zz} 
%                 \\\end{array}\right)\end{math}  Inertia matrix expressed w.r.t. the sensor coordinate frame\\ 
%                 $\ma{J_{c}}$ & Inertia matrix expressed w.r.t. the center of mass 
%                         of the load\\ 
%                 $\ma{J_{kk}}$ & Moment of inertia around the $k$-axis\\ 
%                 $\ma{J_{kl}}$ & Product of inertia w.r.t. the 
%                     $k,l$-plane\\ 
%                 $m$ & Mass of the load\\ 
%                 $\ve{m}$ & Vector denoting the torque exerted by 
%                 the sensor on the load\\ 
%                 $\ve{m_{o}}$ & Torque offset\\ 
%                 $N$ & Variable denoting quantities\\ 
%                 $orth(\ma{X})$ & Orthonormal basis of matrix X\\ 
%                 $p(x)$ & probability that $x$ occurs\\ 
%                 $\ve{p\;^{load,i}_{s}}$ & Point $i$ of the load 
%                 bounding box expressed w.r.t. the sensor frame\\ 
%                 $\ma{P}$ & Error covariance matrix of the Kalman filter 
%                 and RLS-based methods\\ 
%                 $q_{i}$ & Joint angle of joint $i$\\ 
%                 $\dot{q}_{i}$ & Joint angular velocity of joint 
%                 $i$\\ 
%                 $\ddot{q}_{i}$ & Joint angular acceleration of joint 
%                 $i$\\ 
%                 $q_{i,0}$ & Initial joint angle of joint $i$\\ 
%                 $q_{ii}$ & Element of the process noise covariance 
%                 matrix\\ 
%                 $\ma{Q}$ & Process noise covariance matrix\\ 
%                 $\ve{r}$ & Vector pointing from the origin of the base 
%                       reference frame to the center of mass of the 
%                       load\\ 
%                 $^{j}_{i}\!R$ & rotation matrix relating the 
%                 frames $i$ and $j$\\ 
%                 $r_{ii}$ & Element of the measurement noise covariance 
%                 matrix\\ 
%                 $\ma{R}$ & Measurement noise covariance matrix\\ 
%                 $s$ & sample standard deviation\\ 
%                 $t$ & Point in time or period of time\\ 
%                 $t_{acq}$ & Acquisition time of measurements for sensor resets\\ 
%                 $t_{c}$ & control period of the position 
%                 controller\\ 
%                 $t_{max}$ & maximum identification time\\ 
%                 $t_{s}$ & Sampling period\\ 
%                 \end{tabular} 
%                 \end{minipage} 
% \begin{minipage}{16cm} 
%                 {\bf \Large Nomenclature} 
%                 \\\\\\ 
%                 {\bf \large Roman Letters\\\\} 
%                 \begin{tabular}{ll} 
%                 $t_{sta}$ & Period of time after which JR3 sensor 
%                 temperature has stabilized\\ 
%                 $\ma{^{i}_{j}T}$ & homogeneous transformation matrix 
%                 relating frame $i$ and $j$\\ 
%                 $\ma{U}$ & Left singular matrix\\ 
%                 $\ve{v}$ & Measurement noise vector of the Kalman 
%                 filter\\ 
%                 $\ma{V}$ & Right singular matrix\\ 
%                 $\ve{w}$ & Process noise vector of the Kalman 
%                 filter\\ 
%                 $\ma{W}$ & Instrumental variables matrix\\ 
%                 $\ve{x}$ & State of the Kalman filter, the DUKF, and the APF\\ 
%                 $\ve{y}$ & Measurement vector of the Kalman 
%                 filter\\ 
%                 \end{tabular} 
%                 \end{minipage} 
 
\setstretch{1.5}

    \chapter{Einleitung}
    \label{einleitung}
    \pagenumbering{arabic}
    Unsere Aufgabe ist es, eine koordinierte Ballsuche in einen Raum
    zu implementieren. \\\\

Das Ziel der koordinierten Suche besteht darin, dass die im Raum bzw. im
Suchterrain befindlichen mobilen Roboter miteinander das gesuchte Ziel
finden. Diese Suche setzt voraus, dass sich die Roboter kollisionsfrei
im Raum bewegen. Sie d�rfen weder mit Objekten aus der Umgebung, noch
mit anderen Robotern, die ebenfalls auf der Suche nach dem gegebenen
Objekt sind, kollidieren. Zur Umsetzung dieser Aufgabe stanen eine
Ad-hoc-L�sung und schlie�lich eine Server-Client-L�sung zu Auswahl, auf
die die Entscheidung fiel. Dies erm�glicht eine paaralele Entwicklung
von Client und Server, was eine einfachere Aufgabenverteilung im Team
erm�glicht. Ausserdem kann dann auf den Server eine Visualisierung der
Ballsuche erfolgen, indem eine Karte des Raumes, sowie der darin
befindlichen Roboter sowie (nach erfolgreicher Suche) des Balls in der
GUI des Servers angezeigt wird. Der Client ist dann nur noch f�r das
Abfahren der von Server vorgebenen Postionen, sowie die Lokalisierung
im Raum und Ballerkennung in seiner unmittelbaren Umgebung
zust�ndig. \\\\
Zum Erreichen des Ziels standen uns bis zu drei mobile Roboter 
    ,,Pioneer-3DX'' der Firma ,,adept mobilerobots'' zur
    Verf�gung. Dazu konnten wir diverse Libaries des Instituts
    nutzen. Die wichtigsten schon vorhandenen Komponenten waren aber
    der SonarPartikelfilter, sowie die Ballerkennung:
    \begin{itemize}
    \item Der SonarPartikelfilter ging auf ein vorheriges Teamprojekt
      zur�ck. Mit ihm konnten wir die Position des Roboters im Raum
      relativ  genau bestimmen. Diese diente zum einen als
      Basis f�r die Bahnplanung. Zum anderen war dies die Basis zur
      Bestimmung der Position eines erkannten Balls.
    \item Die Ballerkennung konnten wir aus der Bachelorarbeit von
      Tobias Breuer �bernehmen. Sie stelle uns eine API zur Verf�gung,
      wor�ber wir den Ball finden, sowie seine genaue Position
      bestimmen konnten.
    \end{itemize}
Unsere Softwarearchitektur bestand somit aus folgenden Komponenten:
\begin{itemize}
\item Der Server: Er nimmt die Bahnplanung, sowie Kollisionsvermeidung
  vor und steuert bis zu drei Clients. Gleichzeitig dient er als
  Visualisierung des aktuellen Status der Ballsuche.
\item Der Client: Eine Steuersoftware f�r die Roboter. Sie ist daf�r
  zust�ndig, die vom Server vorgebene Bahnplanung umzusetzen. Daf�r
  nutzt er den SonarPartikelfilter, um seine eigene Position im Raum
  zu bestimmen. Au�erdem nimmt er die eigentliche Ballerkennung vor
  und bestimmt aus den dabei erhalten Informationen und der durch den
  Partikelfilter erhaltenen Position die Position des Balles im Raum.es
\end{itemize}
Im n�chsten Abschnitt folgt nun eine genauere Beschreibung unserer Software-Architektur
%Es folgt nun die Beschreibung
%%% Local Variables: 
%%% mode: latex
%%% TeX-master: "template"
%%% End: 
%worum gehts? 

\chapter{Softwarearchitektur}
\label{cha:softwarearchitektur}

\section{Allgemeiner Aufbau}
\label{sec:allgemeiner-aufbau}

Die Software besteht aus zwei Komponenten:
\begin{itemize}
\item Der Server, der die Bahnplanung für die Roboter, sowie die
  Visualisierung der Ballsuche übernimmt.
\item Der Client, der die Ballerkennung, sowie Positionsbestimmung
  vornimmt. Dieser ist ausserdem dafür verantwortlich, den Roboter zu
  steuern. Dazu lässt der Client den Roboter die durch den Server vorgegebenen
  Punkte der geplanten Bahn anfahren. 
\end{itemize}
%\section{Server}
%\label{sec:server}
\def\bilderpath{bilder/}

\section{Server}\label{serv:Server}

Der Server dient zum einen als zentraler Netzwerk-Einstiegspunkt für alle Clients, und zum anderen auch für die Koordination der Suche und Bahnplanung der verbundenen mobilen Roboter auf der bereitgestellten Karte des Raumes. Die Schnittstelle zwischen Server und Client wurde so gering wie möglich gehalten um sowohl den Server als auch den Client so dynamisch wie möglich verweden zu können.

\subsection{Grafische Benutzeroberfläche (GUI)}\label{serv:archGui}
Zur Visualisierung der im Projekt implementieren Funktionen wurde
mittels der objektorientierten Klassenbibliothek der Microsoft
Foundation Class (MFC) eine Benutzeroberfläche erstellt. Abgeleitet
von der Klasse "CFormView" dient das äußere Fenster als Gerüst für die
benötigten Aktions- und Einstellungselemente, sowie für die
Darstellung der Karte und des Protokollierungsfensters.  Die GUI wird
näher im Abschnitt \ref{serv:gui} ab Seite \pageref{serv:gui} beschrieben.
\subsection{Netzwerkverbindung}\label{serv:netzwerkverbindung}
Für die Kommunikation zwischen Server und Client dient eine TCP-IP-Socket-Verbindung. Wie in Abbildung \ref{serv:fig:netzwerkverbindung} zu sehen, kann nach individuellen Initialisierungsschritten von Server von Client eine Datenverbindung aufgabaut werden, in der die entsprechenden Nutzdaten für die Robobter und für die Serversoftware übertragen wird.

\begin{figure}[h]
	\centering	
	\includegraphics[width=8cm]{\bilderpath server_client_verbindung.png}
	\caption{Server-Client Kommunikationsverbindung.}
	\label{serv:fig:netzwerkverbindung}
\end{figure}

\subsubsection{Verbindungsaufbau}

Zum erstellen einer Socket-Verbindung muss zunächst ein Socket, unter Angabe von Protokoll und Übertragungsart, erstellt werden. Anschließen wird an diesen Socket die Inforamtion über den Service angeheftet. Darunter fallen Kommunikationsart (bsp. TCP oder UDP), Adressdomain (host:port oder UNIX Pathname) und der verwendete Port. Als letzter Initialisierungsschritt, muss die Anzahl der Verbindungen, die der Socket zulässt, angegeben werden. Ist die Initialisierung erfolgt, kann der Socket auf eine Verbindung aus dem Netzwerk warten. Dies geschieht mit der Funktion {\it accept(...)}, die das Programm bzw. den Thread blockiert, bis eine Verbindung eingegangen wird. Für weitere Verbindungen muss die {\it accept(..)}-Funktion erneut aufgerufen werden. Der Rückgabewert der Funktion entspricht der Socketnummer und die Variable die als Referenz der Funktion übergebenden wird, enthält nach der eingehenden Verbindung die Informationen über den verbundenen Client. In der Auflistung \ref{serv:lst:serverconn} werden die essentiellen Befehle für die Verbindung des Servers zusammengefasst.

\begin{lstlisting}[frame=tb,captionpos=b,caption=Socket Connection - Server., label=serv:lst:serverconn]
//Zusammenfassung der Implementierung aus der SocketConnection.cpp
#pragma comment(lib,"WSOCK32.LIB")

SOCKET cSocket;
sockaddr_in service;
int tempsize;
struct sockaddr_in tempstr;
tempsize = sizeof(tempstr);	

service.sin_family = AF_INET; 								//Domain								
service.sin_addr.s_addr = htonl(INADDR_ANY);	//initialisietung der Adresse
service.sin_port = htons(7410);								//Port

//Socket erstellen
cSocket =  socket(AF_INET,SOCK_STREAM,IPPROTO_TCP); 			
//Service anheften
bind(cSocket, (SOCKADDR*)&service, sizeof(service)); 
//Anzahl der Verbindung (CONNECTION_COUNT) angeben
listen(cSocket,CONNECTION_COUNT);														

//Auf Verbindung warten
int acception = accept(cSocket, (struct sockaddr*)&tempstr, &tempsize); 

\end{lstlisting}

Um eine Verbindung aufzubauen muss der Client zunächst, wie der Server, einen Socket erstellen. Anschließend kann direkt durch Angabe der Adressdomain, der IP-Adresse und dem Port eine Verbindung mit dem wartenden Server eingegangen werden, wie in Listing \ref{serv:lst:clientconn} gezeigt. Während des Versuches eine Verbindung mit dem Server einzugehen wird das Programm bzw. der Thread geblockt bis die Verbindung zustande gekommen ist oder ein interner Timeout den Verbindungsversuch terminiert.

\begin{lstlisting}[frame=tb,captionpos=b,caption=Socket Connection - Client., label=serv:lst:clientconn]
//Zusammenfassung der Implementierung aus der Connection.cpp das Clients

#pragma comment(lib,"WSOCK32.LIB")
int m_socket;
struct sockaddr_in m_serverAddress;

m_serverAddress.sin_addr.S_un.S_addr = inet_addr('192.168.1.1');	//Server-IP
m_serverAddress.sin_port = htons(7410);						//Port
m_serverAddress.sin_family = AF_INET;							//Domain	

//Socket erstellen
m_socket = socket(AF_INET,SOCK_STREAM,IPPROTO_TCP);
//Mit Server verbinden
int check = connect(m_socket, (sockaddr*)&m_serverAddress,
													sizeof(m_serverAddress)) ;
\end{lstlisting}

\subsubsection{Datenaustausch}
Zum Senden und Empfangen von Daten wurde eine Reihenfolge von Aktionen festgelegt, da ein Paket, das gesendet wurde, auch von der Gegenseite empfangen werden muss. Dieser Ablauf wird in Abbildung \ref{serv:fig:datenaustausch} dargestellt. Neben den Nutzdaten für und über die mobilen Roboter werden desweiteren ACK's gesendet, die ein erfolgreiches Empfangen signalisieren. Diese Implementierung wurde zunächst für die ersten Testreihen genutzt und sollte im späteren Verlauf entfernt werden. Jedoch erwiesen diese sich als nützlich zur Fehlererkennnung bei einer Übergebe falscher Daten und sind aus Programmstabilitätsgründen erhalten geblieben.

\begin{figure}[h]
	\centering	
	\includegraphics[width=8cm]{\bilderpath server_client_datenaustausch.png}
	\caption{Server-Client Datenaustausch.}
	\label{serv:fig:datenaustausch}
\end{figure}

Der Austausch der Informationen erfolgt in erstellten Datenstrukturen, die in Kapitel \ref{serv:datenstrukturen} erläutert werden. Diese angelegten und mit Informationen erfüllten Strukturen werden zu einem Zeitpunkt vom Server bzw. Client gesendet und vom jeweils anderen Kommunikationspartner empfangen. Dies erfolgt über die in Listing \ref{serv:lst:sendenempfangen} angegebenen Funktionen. Mittels der Übergabe des Sockets der Verbingung, werden die Daten zwischen den korrekten Kommunikationspartnern ausgetauscht. Sowohl die zu sendenden Daten als auch die zu empfangenden Daten befinden sich in der Struktur, die als Referenz an die Funktion übergeben wird. Nach dem Ausführen der Funktion wird ein ganzzahliger Wert zurückgegeben. Dieser entspricht der Anzahl der übermittelten Bytes. Sollten keine Daten übertragen werden gibt die Funktion einen negativen Wert zurück.

\begin{lstlisting}[frame=tb,captionpos=b,caption=Senden und Empfangen., label=serv:lst:sendenempfangen]
int check;
struct Information info;

//Daten senden (die Variable muss Daten enthalten)
check = send(getSocket(),(char*)info,sizeof(Information),0);

//Daten empfangen
check = recv(getSocket(),(char*)info,sizeof(Information),0);
\end{lstlisting}

\subsubsection{Datenstrukturen für den Datenaustausch}\label{serv:datenstrukturen}

Die Datenstrukturen für den Datenaustausch konzentrieren sich auf die wesentlichen Informationen, die zwischen dem Server und einem Client ausgetauscht werden müssen. Ziel war es eine möglichst kleine und allgemeine Schnittstelle zu konstruieren um eine Verwendung mit weiteren Robotern nicht auszuschließen. Beispielsweise wäre die Übertragung der Ultraschallsensordaten des Roboters, nicht durch Roboter anderer Modelle möglich und wurde daher auch nicht in die Übertragungsstruktur aufgenommen. Die in Listing \ref{serv:lst:str:robotinfo}, \ref{serv:lst:str:searchinfo} und \ref{serv:lst:str:pathinfo}  zu sehenden Strukturen, sind die Informationsträger, die zwischen dem Server und dem Client übertragen werden.

\begin{lstlisting}[frame=tb,captionpos=b,caption=Datenstruktur RobotInformation., label=serv:lst:str:robotinfo]
struct RobotInformation
{			
	Point pos;					//Aktuelle Standordsinformationen des Roboters
	int angle;					//Angle in GRAD!!!

	bool ballDetected;	//Ball gefunden?
	Point ballPos;			//Position des gefundenen Balles

	bool needPathTable; //Neuen Pfad anfordern

	bool pointReached; 	//Ein Punkt wurde erreicht
};
\end{lstlisting}

Die RobotInformation-Struktur aus \ref{serv:lst:str:robotinfo} beinhaltet Informationen über den aktuellen Status des Roboters. Dazu gehören die Koordinaten an denen sich der mobile Roboter zum Übertragungszeitpunkt befindet. Zur Visualisierung auf der Karte des Servers wird außerdem noch die Ausrichtung des Roboters übermittelt, die als Winkel der x-Achse und der Blickrichtung definiert ist. Sollte der Roboter einen Ball im Raum detektieren, wird diese Information inklusive der Ballposition Übertragen. Die Ballposition ist dabei für die Darstellung auf der Karte nötig.\\ Während der Roboter die vom Server generierten Wege abfährt, werden zwei Signalisierungstypen benötigt. Zum einen wird, für die Darstellung der Karte des Servers, signalisiert, dass ein Punkt der Liste erreicht wurde, der beim nächsten Zeichnen der Karte gelöscht werden kann und zum anderen wird signalisiert, dass ein neuer Pfad benötigt wird, sobald alle Punkte abgearbeitet wurden. 

\begin{lstlisting}[frame=tb,captionpos=b,caption={Datenstruktur  SearchInformation.}, label=serv:lst:str:searchinfo]
struct SearchInformation
{
	int transmissionPathTableSize;	//Pfadl"ange von PathInformation

	bool ballDetected;							//Ball wurde gefunden
	
	bool pause;											//Suche Aktiviert/Deaktiviert
};
\end{lstlisting}

In der Struktur SearchInformation aus \ref{serv:lst:str:searchinfo} werden Informationen vom Server an den mobilen Client gesendet. Diese Informationen signalisieren dem Roboter den Zustand der Suche. Pausiert die Suche, so sollte der Roboter sich nicht im Raum bewegen. Andersfalls läuft sie Suche und der Roboter darf die Fahrt aufnehmen. Wurde der Ball von einem der in der Suche verwendeten Roboter gefunden, ist die Suche beendet und damit können auch alle Roboter ihre Suche beenden. Dies wird durch die Variable ballDetected signalisiert. Desweiteren muss die Größe des zu übertragenden Pfades mitgeteilt werden, denn zur Übertragung von Daten muss stets die Größe der zu übermittelnden Daten angegeben werden.

\begin{lstlisting}[frame=tb,captionpos=b,caption=Datenstruktur PathInformation., label=serv:lst:str:pathinfo]
struct PathInformation
{
	std::vector<Point> pathList;		//Liste des Pfades
};
\end{lstlisting}

Die PathInformation aus \ref{serv:lst:str:pathinfo} Übermittelt die vom Server generierten Pfade, die durch den Bahnplaner enstanden sind. Dieser Vorgang erfolgt, sobald der Client eine neue Liste von Wegpunkten anfordert. \\

Aus den Strukturen der Listings \ref{serv:lst:str:robotinfo}, \ref{serv:lst:str:searchinfo} und \ref{serv:lst:str:pathinfo} ergibt sich schließlich die in Abbildung \ref{serv:fig:schnittstelle} dargstellte Schnittstelle, die im  definierten Regeltakt zwischen Server und Client übertragen wird.

\begin{figure}[h]
	\centering	
	\includegraphics[width=8cm]{\bilderpath server_client_schnittstelle.png}
	\caption{Schnittstelle zwischen Server und Client.}
	\label{serv:fig:schnittstelle}
\end{figure}

\subsection{Testclient}\label{serv:testclient}

Zum Testen der Anwendung wurde eine Testclientumgebung implementiert. Diese dient zur Prüfung der Netzwerkverbindung und Simulation der mobilen Roboter ohne die Verwendung eines realen Roboters. Die Client-Konsolen-Applikation verbindet sich nach dem starten mit dem Server und füllt die Daten der Schnittstelle mit fiktiven Informationen. Um das Verhalten eines Roboters zu simulieren, wird, bei einer gestarteten Suche, in jedem Zeitschritt die benötigte Orientierung ermittelt und in diese Richtung eine vordefinierte Länge zurückgelegt. Kollisionen und anderen unerwarteten Vorkommnisse werden nicht berücksichtigt.

\subsection{Karte - Wabenkarte und Roboterscan}\label{serv:maps}

Zur Navigation der Roboter muss der Raum, in dem sie sich bewegen sollen, bekannt sein. Dazu sind in einer Karte in Positionen von Hindernissen (z.B. Wände) und freie Flächen hinterlegt. Die Punkte und die dazugehörigen Informationen sind in einer Bitmap-Bilddatei kodiert. Sie besitzt eine Grauwerttiefe von 8 Bit und kann damit 256 verschiedene Informationen über einen Punkt enthalten. Die Datei, die zur aktuellen Beschaffenheit des Raumes passt muss in das System geladen werden und in einem $n\times m$ Array gespeichert werden, um weitere Berechnungen im Programm durchzuführen. Dabei entspricht $m$ der Höhe und $n$ der Breite des Bildes in Pixel. Für den Roboter bedeutet dies, dass dieser sich nur auf den Koordinaten im Raum bewegen darf, die auf der Karte einen Wert von 255 besitzen. Alle anderen Werte bzw. Punkte sind nicht befahrbar, da sie ein Hindernis darstellen. \\
Da sich schnell herausstellt, dass die Berechnung auf einer Karte mit einer Höhe und Breite von jeweils 800 Pixeln mindestens 640.000 Rechenschritte benötigt und dadurch das Berechnen eines Pfades viel Zeit in Anspruch nimmt, kam die Idee den Vorgang zu beschleunigen. \\
Die Lösung ist die Karte zu verkleinern, da eine so hohe Auflösung des Raumes nicht benötigt wird. Dazu wird eine neue Karte erstellt, die in der Höhe und Breite um den Faktor $F$ kleiner ist. Damit wird die Rechenzeit um den Faktor $\frac{1}{F^2}$ beschleunigt. Die Pixel der neuen Karte werden im Programm als Wabe (in Anlehnung zu Bienenwabe, engl. "`comb"') und die Karte als Wabenkarte bezeichnet. Eine Wabe repräsentiert die Information von $F^2$ Pixeln. Um eine maximale Sicherheit zu erhalten, ist eine Wabe für den Roboter nur befahrbar, wenn alle korrespondierenden Pixel der Originalen Karte für den Roboter befahrbar sind. Das stark vergrößerte Ergebnis der  Umwandlung einer Roboterscankarte (Abbildung \ref{serv:fig:roboterscan}) zu einer Wabenkarte wird in Abbildung \ref{serv:fig:wabenkarte}, wobei die grünen Kästen den Rand der Wabe markieren und in der Datenstrukur im System nicht vorhanden sind. Die GUI ermöglicht, wie in Kapitel \ref{serv:Gui} beschrieben, das wechseln der Ansichten für die Darstellung der Karte.

\begin{figure}[H]
	\begin{minipage}[b]{8cm} 	
	\centering
	\includegraphics[width=7cm]{\bilderpath server_robotscanmitgitter.png}
	\caption{Roboterscanausschnitt.}
	\label{serv:fig:roboterscan}
	\end{minipage}
	\hfill	
	\begin{minipage}[b]{8cm} 
	\centering	
	\includegraphics[width=7cm]{\bilderpath server_combgitter.png}
	\caption{Resultierende Wabenkarte.}
	\label{serv:fig:wabenkarte}
	\end{minipage}
\end{figure}

\subsection{Der Regeltakt}\label{serv:regeltakt}

Mit dem Verbindungsaufbau wird für jeden Roboter ein separater Thread gestartet. Nach der eingehenden Verbindung verweilt jeder Thread in einer Dauerschleife, die durch den Benutzer beendet werden kann und damit die Verbindung zwischen den Geräten trennt. Die Dauerschleife terminiert ebenfalls, wenn die Verbindung zum Roboter, durch einen Verbindungsabbruch oder beenden des Clients, getrennt wird. Wird die Schleife beendet, so wird auch der Thread beendet und der Server ist bereit für eine erneute Verbindung mit einem Roboter. \\

Während des Schleifendurchlaufs werden im Regeltakt alle Operationen durchgeführt, um die erforderlichen Daten, die der Roboter benötigt, breitzustellen und zu übertragen, sowie alle Informationen, die der Server benötigt, für die Darstellung, Speicherung und Weiterverarbeitung, bereitzustellen und zu setzen. Bevor die Schleife beginnt werden alle notwendigen Variabeln, darunter fallen auch die RobotInformation und SearchInformation, initalisiert und anschließend der Begin der Dauerschleife eingeleitet. Durch die Initialisierung wird festgelegt, dass der Client noch keinen neuen Pfad benötigt, deshalb wird im ersten Durchlauf der Schleife die Ermittlung eines Punktes durch die Koordination, die generierung des Pfads der Bahnplanung und Bestimmung kollisionsfreier Fahrten der Kollisionsvermeidung  übersprungen. Dies führt direkt zur ersten Nutzdatenübertragung zwischen Server und Client. Dabei erhält der Server die Positionsdaten des Roboters. Diese werden direkt in den entsprechenden Variablen gespeichert, um bei erneutem Zeichnen der Karte die aktuelle Position des Roboters darstellen zu können. An dieser Stelle endet die Schleife und der Prozess beginnt von vorne. Vor dem Neubeginn der Schleife wir vorweg noch eine Sleep eingefügt, der die Threadkontrolle abgibt und dem System erlaubt während dieser Wartezeit andere Operationen durchzuführen. Sollte der Client einen neuen Pfad angefordert haben, wird dieser, im nächsten Durchlauf, mittels der Koordination, Bahnplanung und Kollisionsvermeidung generiert (Kapitel \ref{serv:kbk}). Der erstellte Pfad sowie die Länge des zu übergebenden Pfades und der Status der Suche werden in den entsprechenden Strukturen gespeichert und anschließend übermittelt. Schließlich ergibt sich aus dem schematischen Programmcode aus \ref{serv:lst:thread} der in Abbildung \ref{serv:fig:regeltakt} dargestellte Verlauf für den Regeltakt.\\

\begin{lstlisting}[frame=tb,captionpos=b,caption=Thread des Regeltaktes., label=serv:lst:thread]
#define SLEEPTIME 500
Ponint point;

while(holdConnection) 					//Dauerschleife
{	
	if(needPathTable)
	{
		//Punkt durch Koordination ermitteln
		point = GetNextPoint(pSelfRob->robotId);
		
		//Pfad durch Bahnplanung ermitteln
		CreateNextPath(robotId, robot->position ,point, &internalPathList);
		
		//Kollisionen verhindern
		avCollision->savePath(robot->position, &internalPathList);
	} 	
	
	//Nutzdaten"ubertragung
	connection->Transmit(robotInformation, searchInformation, internalPathList);
	
	//Abgabe der Threadkontrolle
	Sleep(SLEEPTIME);
	
	//Abbruch einleiden
	holdConnection = changeConnectionStatus();
}

\end{lstlisting}


\begin{figure}[h]
	\centering	
	\includegraphics[width=15cm]{\bilderpath server_regeltakt.png}
	\caption{Programmablauf des Roboter-Threads.}
	\label{serv:fig:regeltakt}
\end{figure}

Die Länge der Wartezeit nach einer Schleife kann vom Benutzer im Programm festgelegt werden. Während der Versuche hat sich die Wartezeit von 500ms als gutes Maß bewährt. Für die Zeit $t_{Takt}$ in der eine Übertragung pro Thread erfolgt ergibt sich demnach aus: 
\begin{equation}
t_{Takt} = t_{Operationen}+t_{Warten}
\end{equation}
mit $t_{Operationen}$ als die Zeit die benötigt wird um die Operationen Koordination, Bahnplanung und Kollisionsvermeidung durchzuführen und $t_{Warten}$ als feste Wartezeit, die vom Benutzer angegeben wird. Da die Zeit $t_{Operationen}$ für jeden Takt unterschiedlich ist, ist folglich auch die Zeit des Regeltakts unregelmäßig. 

\subsection{Koordination, Bahnplanung und Kollisionsvermeidung}\label{serv:kbk}
\subsubsection{Koordination}
Der Weg, den die Roboter während der Suche zurücklegen, wird durch die Koordination, Bahnplanung und Kollisionsverbeidung berechnet und festgelegt. Die Koordination ermittelt einen neuen Punkt im Raum, den der Roboter als nächstes anfahren soll. Dies geschieht global. Das bedeutet, dass alle Roboter gemeinsam nach dem gleichen Koordinationsalgorithmus fahren. Strategie für eine Koordination wäre beispielsweise eine Discovery-Map, in der die Punkte eingetragen sind, die die Roboter schon besucht bzw. erkundet haben und die Punkte, die noch nicht erkunden wurden müssen als nächstes angefahren werden. Zurzeit sind zwei sehr einfache Koordination-Strategien implementiert, da komplexe Strategien in dem zur Verfügung stehenden Suchterrain nicht nötig sind. In der ersten Variante werden Fest eingespeicherte Punkte im Raum angefahren, die jedoch für jede Karte neu eingetragen werden müssen. Die andere Strategie basiert auf Zufallspunkten, die währende der Laufzeit erstellt werden. Dabei wird eine zufällige Koordinate erstellt und überprüft, ob diese auch auf einem befahrbaren Punkt liegt. Sollte der generierte Punkt in einem Hindernis liegen, wird so lange ein neuer Punkt generiert, bis der Punkt nicht im Hindernis liegt.\\
\subsubsection{Bahnplanung}
Mit dem Punkt der Koordination wird das neue Ziel des Roboters generiert, das als nächstes erreicht werden muss. Dazu berechnet die Bahnplanung einen Weg durch den Raum, den der Roboter abfahren soll. Im Gegensatz zur Koordination, kann jeder Roboter eine unterschiedliche Bahnplanung besitzen. Dazu stehen verschiedene Strategien zur Auswahl. Beispielsweise ein RRT (Rapidly-exploring random tree), der zufällig Punkte im Raum generiert und diese zu einem Weg vom Roboter zum Zielpunkt verbindet. Ziel ist es durch die Bahnplanung eine Liste von Punkten zu erhalten, die der Roboter kollisionsfrei durch Rotation und Vorwärtsbewegung, erreichen muss. In der aktuellen Version wurde eine selbst entwickelte Methode implementiert, die "`Fast Sweep 4"' genannt wurde (vermutlich bereits unter einem anderen Namen bekannt). Diese Methode startet am Punkt an dem sich der Roboter befindet und breitet sich von Schritt zu Schritt über die Karte aus, bis das Ziel erreicht wurde. Dabei werden in jedem Schritt die Entfernungen zum Roboter in den leeren anliegenden Zellen der Karte gespeichert, die beim erreichen des Punktes, durch Rückwärtslaufen, zu einem Pfad erstellt werden. Als Ergebnis dieser Funktion entsteht eine Liste von Punkten, die durch eine Interpolation die unnötigen Zwischenpunkte entfernt und daraus die benötigte Pfad-Liste resultiert.\\
\subsubsection{Kollisionsvermeidung}
Neben der Kollision, die mit der Umgebung entstehen kann, muss bei einer Suche mit mehreren Robotern auch sichergestellt werden, dass keine Kollision untereinander entsteht. Dafür stand beispielsweise eine Strategie zur Verfügung, die einer Methode der Bahntechnik ähnelt. Dazu wird nach dem Erreichen eines Punktes überprüft, ob der nächste Fahrabschnitt sicher ist und eine Fahrstraße gestellt werden kann, die zu dem Zeitpunkt von keinem anderen Roboter genutzt werden kann. Die implementierte Methode sperrt nach der Bahnplanung den kompletten Raum, der von einem Roboter abgefahren werden soll, in dem die Punkt der Karte als Hindernis für die Bahnplanung der andren Roboter eingestellt werden. Erreicht der Roboter die Punkte der Bahnplanung, wird der bereits bewältigte Raum wieder für alle Roboter freigegeben.

\subsection{Erweiterung von Koordination und Bahnplanung}

Die Struktur des Programms ist so Konstruiert, dass durch ein hinzufügen weniger Programmcodezeilen neue Koordinationsalgorithmen und Bahnplanungsalgorithmen aufgenommen werden können. Koordinationen, die hinzugefügt werden sollen sind Klassen, die von der Klasse "`Coordination"' abgeleitet werden müssen. Das Ableiten bewirkt den Zwang die virtualle Methode {\it GetNextPoint(..)} zu implementieren, die im Programmverlauf vom Roboter aufgerufen wird und den neuen Zielpunkt für den Roboter ermittelt und zurückgibt. Zur Berechnung des Punktes steht der Koordination lediglich die Karte des Raumes zur Verfügung. Das Header-Template einer solchen Klasse ist in Listing \ref{serv:lst:koordtemp} dargestellt.

\begin{lstlisting}[frame=tb,captionpos=b,caption=Koordinationstemplate., label=serv:lst:koordtemp]
#pragma once
#include "coordination.h"


class Coordination_None : public Coordination
{
public:
    Coordination_None(void);
    ~Coordination_None(void);

    Point GetNextPoint(int robotId);
};
\end{lstlisting}

Die Integration einer neuen Bahnplanung ist identisch zu der Integration einer Koordination. Bahnplanungen müssen von der Klasse "`Pathplan"' abgeleitet werden, damit sichergestellt wird, dass die Funktion {\it CreateNextPath(..)} vorhanden ist. Diese berechnet durch die Angabe von Startpunkt und Zielpunkt eine neue List von Wegpunkten für den Roboter. Die Liste befindet sich nach der Funktion in der übergebenen Referenz "`InternalPathInformation"'. Listing \ref{serv:lst:bahnplantemp} entspricht dem Header-Template der Klasse.

\begin{lstlisting}[frame=tb,captionpos=b,caption=Bahnplantemplate., label=serv:lst:bahnplantemp]
#pragma once
#include "Pathplan.h"

class Pathplan_None : public Pathplan
{
public:
    Pathplan_None();
    ~Pathplan_None();

    int CreateNextPath(int robotId, Point start, Point end, InternalPathInformation *ipi);
};
\end{lstlisting}

Der angezeigte Name der Koordiantion und Bahnplanung sehen in den Variablen "`coordinationName"' und "`pathplanName"' die in den jeweiligen Basisklassen enthalten sind. Sind die Klassen im Programm integriert, müssen diese noch im Programm angemeldet werden. Dies geschieht durch das Inkludieren der Header-Dateien in der "`stdafx.h"' wie in Listing \ref{serv:lst:anmeldungstda} und das Erstellen der Instanz in der Datei "`PathManager.cpp"', wie es in Listing \ref{serv:lst:anmeldungpm} gezeigt wird.

\begin{lstlisting}[frame=tb,captionpos=b,caption=Anmeldung der Klassen in stdafx.h., label=serv:lst:anmeldungstda]
//UsernInput:------//User muss seine neu erstellte *.h Datei hier einf"ugen!                                    
#include "Coordination.h"
#include "Coordination_None.h"
#include "Coordination_Random.h"
#include "Coordination_Fixpoint.h"

#include "Pathplan.h"
#include "Pathplan_None.h"
#include "Pathplan_FastSweep4.h"
\end{lstlisting}

\begin{lstlisting}[frame=tb,mathescape=true,captionpos=b,caption=Anmeldung der Klassen in PathManager.cpp., label=serv:lst:anmeldungpm]

//UserInput: User muss die Anzal der Coordinationen Manuell eintragen                                          
$\color{red}\textbf{numberOfCoordinations = 3;}$
coord = new Coordination*[numberOfCoordinations];
for(int i=0; i<numberOfCoordinations; i++)
{
    coord[i] = NULL;
}

//UsernInput: Koordinationsklassen hinzuf"ugen                
$\color{red}\textbf{coord[0] = new Coordination\_None();}$
coord[1] = new Coordination_Fixpoint();
coord[2] = new Coordination_Random();  

//UserInput: User muss die Anzal der Bahnplaner Manuell eintragen                                          
$\color{red}\textbf{numberOfPathplaner = 2;}$
pathplan = new Pathplan*[numberOfPathplaner];
for(int i=0; i<numberOfPathplaner; i++)
{
    pathplan[i] = NULL;
}
//UsernInput: Bahnplanung hinzuf"ugen
$\color{red}\textbf{pathplan[0] = new Pathplan\_None();}$
pathplan[1] = new Pathplan_FastSweep4();

\end{lstlisting}
%%% Local Variables: 
%%% mode: latex
%%% TeX-master: "template"
%%% End: 

\chapter{Der Client}
Der Client in unserer Client-Server-Architektur ist für das Anfahren der
einzelnen Routenpunkte und die Erkennung des Balls zuständig. Dazu haben
wir für die Lokalisierung die uns zu Verfügung gestellte Implementierung
eines Sonar-Partikelfilters und für die Ballerkennung die Implementierung
aus einer Bachelorarbeit von Tobias Breuer verwendet.

Am Partikelfilter haben wir ein paar Änderungen vorgenommen. Es gab keine
Möglichkeit, von außerhalb des Filters auf die Koordinaten und die
Ausrichtung des Roboters zuzugreifen. Dazu wurden die Methoden StartFilter
und findBestParticle so erweitert, dass die aktuelle Position des Roboters
und seine Ausrichtung als Zeiger auf das erste Element eines double-Arrays
([x,y,Theta, Partikelwsk.]) zurückgegeben wird. Da dieses in
findBestParticle dynamisch mit new alloziiert wird, muss es, wenn es nicht
mehr beötigt wird, mit delete[] gelöscht werden. Außerdem ist nun möglich
die Anzahl der Sonare von außerhalb des Filters ohne neukompillieren
einzustellen. Die Visualisierungen wurden aus Performancegründen entweder
entfernt oder deaktiviert. Zudem wurde für eine einfachere Auswertung und
weil die Partikelvisualisierung öfter zu Abstürzen unseres Programms
geführt hat der Filter um eine Funktion erweitert, die, wenn in der
particleSet.cpp takeADump true ist, bei jedem Filtern alle Partikel als
Tripel aus x,y und der Partikelwsk. in eine Datei schreibt. Die Datei hat
dabei als Prefix eine Zahl im Namen, die bei 0 beginnend nach jedem Filtern
 um 1 erhöht wird. Passend dazu wurde ein Visualisierer geschrieben, der
die Partikelmenge grafisch darstellt, auf den noch später eingegangen wird.

Der Client selber arbeitet folgendermaßen:

Beim Start wird die Datei config.cfg eingelesen, wenn diese nicht gefunden
wird, nutzt der Client die fest eingestellten Standardwerte. Die config.cfg
hat folgende Struktur:

\begin{lstlisting}
#Kommentarzeile (wird ignoriert)
Sonaranzahl als int
#Kommentarzeile (wird ignoriert)
COM-Port, an dem der Roboter angeschlossen ist, als String
#Kommentarzeile (wird ignoriert)
Schrittweite fuer die Anlerndrehungen als int in Grad
#Kommentarzeile (wird ignoriert)
Anzahl der 360Grad Drehungen als int
#Kommentarzeile (wird ignoriert)
Server-IP als String
#Kommentarzeile (wird ignoriert)
Pixel in der Karte pro mm als double
\end{lstlisting}

Ein Beispielkonfiguration für den Fahrstuhlvorraum wäre z. B.
\begin{lstlisting}
#Anz Sensoren
8
#COM Port
COM5
#Schrittweite fuer die Anlerndrehungen
15
#Anzahl der 360Grad Drehungen
1
#Server-IP
134.169.36.241
#Pixel pro mm
0.05081
\end{lstlisting}
Die Standardwerte sind (16, COM3, 45, 0, 127.0.0.1, 1).

Danach wird die Verbindung zum Roboter hergestellt und der Partikelfilter
mit 10000 Partikeln initialisiert und das erste Mal gefiltert. Danach
erfolgt die konfigurierte Anzahl an Drehungen in der konfigurierten
Schrittweite um die erste Positionierung zu verbessern. Nach jedem
Drehschritt wird einmal gefiltert. Nach den Drehungen wird noch einmal
gefiltert und der Rückgabewert als Startposition genommen. Erst jetzt wird
die Verbindung zum Server hergestellt und die Roboterposition übermittelt,
sowie ein Thread gestartet, der regelmäßig die RobotInformation,
PathInformation und SearchInformation mit dem Server abgleicht. Der Client
wartet dann erstmal so lange, bis der Server in der SearchInformation pause
 = false sendet. Sobald pause = true, wird ein weiterer Thread zur
Ballerkennung gestartet und der Client geht in seine Fahrschleife über, die
 erst endet, wenn der Ball gefunden wurde oder stoppt, wenn pause = true.
 Der Client wartet dann auf den nächsten anzufahrenden Punkt vom Server,
sobald er diesen erhalten hat fährt der Roboter den Punkt an.

Dies geschieht schrittweise:\\
Zuerst wird die Richtung und der Drehwinkel der Drehung für eine
Ausrichtung zum Punkt berechnet. Dazu wird der Winkel zwischen dem
Ausrichtungsvektor des Roboters und dem Punkt berechnet und dann geprüft,
ob der Vektor zwischen Roboterposition und dem Punkt um den ermittelten
Winkel im Uhrzeigersinn oder gegen den Uhrzeigersinn von der
Roboterausrichtung aus liegt um die Drehrichtung zu ermitteln. Dann wird
der Roboter entprechend der ermittelten Werte zum Punkt hin gedreht.
Dann wird, falls pause in der SearchInformation auf true ist, solange
gewartet, bis pause wieder false ist. Danach wird
drivingTime * driveLengthWeight ms gefahren, wobei drivingTime die
Fahrtzeit pro Schritt ist und driveLengthWeight ein Anpassungsfaktor der
Fahrzeit ist für den Fall, dass die Reststrecke in weniger als drivingTime
zurückgelegt werden kann. Dazu wird aus den gefahrenen Strecken in jedem
Fahrtschritt und der jeweiligen Fahrtzeit die Strecke in px pro ms
berechnet und über die Schritte der exponentiell geglätteter Mittelwert
berechnet, woraus dann driveLengthWeight für die Reststrecke berechnet wird
, wenn die zu klein für die normale drivingTime ist.

Dann wird eine neue Ausrichtung zum Punkt berechnet und mit der aktuellen
 verglichen, verkürzt sich dadurch die Entfernung zum Ziel, wird der
Roboter neu ausgerichtet, ansonsten wird die alte Ausrichtung beibehalten.
Danach wird wieder eine Teilstrecke gefahren usw. bis sich dem Punkt auf
einen Abstand < delta\_pos angenährt wurde.
Danach wird gewartet, bis der Server das erreichen des Punktes registriert
hat und ein neuer, anzufahrender punkt übermittelt wurde.







%%% Local Variables: 
%%% mode: latex
%%% TeX-master: "template"
%%% End: 


\chapter{Erstellung der Software}
\label{cha:erstellung}
In diesen Abschnitt beschreiben wir den Erstellungsprozess der
Software. Dazu beschreiben wir erst die Struktur des Projektes, den
Aufbau des Buildsystems für den Client und schließlich den Erstellungsprozess für
Client und Server.
% und am Ende schließlich
%die Benutzung von Client und Server.


\section{Aufbau des Projektbaums}
\label{sec:aufbau_projektbaum}
Der aus dem SVN ausgecheckte Projektbaum  ist wie folgt aufgebaut:\\% \\\\
%\\\\\parbox{1.0\linewidth}{
%{1.0\linewidth}
%\begin{minipage}{1.0\textwidth}

\begin{nofloat}{figure}%{l}{1\textwidth}
\centering
\begin{tikzpicture}
%\tikz 
[font=\footnotesize,
       grow=right, level 1/.style={sibling distance=3.5em},
                   level 2/.style={sibling distance=1em}, level distance=3cm]
  \node {TP2010\_11} % root
     child { node {binaries}}
     child {node {client}}
   child{node {doc} }
   child{node {PioneerMRServer}
     }
  child{node {Visualisierung}
    }; %
\end{tikzpicture}
\caption{Projektbaum nach svn checkout}  
\end{nofloat}
%\end{minipage}
%\newpage 
%\begin{minipage}{1.0\textwidth}
Da beim Client es nötig war, diverse Libaries des Institutes für
Robotik und Prozessinformatik (IRP) der technischen Universität
Braunschweig sowie den Sonar-Partikelfilter und die Ballerkennung zu
integrieren, wurde dazu eine besondere Buildumgebung auf Basis von
CMake geschaffen. Sie
wird im Abschnitt \ref{cha:integr-best-proj} ab Seite
\pageref{cha:integr-best-proj} beschrieben.   

%\parbox{\textwidth}{
Einen ersten Überblick
über die Struktur des Ordners ,,client'' gibt folgende Baumübersicht:\\
%\begin{minipage}{1.0\linewidth}
%\end{minipage}
%\ \\

\begin{nofloat}{figure}%{i}{\textwidth}
\centering
\begin{tikzpicture}
%\tikz 
[font=\footnotesize,
       grow=right, level 1/.style={sibling distance=3.5em},
                   level 2/.style={sibling distance=1em}, level distance=3cm]
  \node {client} % root
     child { node {build}}
     child {node {dlls}}
   child{node {include} }
   child{node {lib}
     }
  child{node {src}
    }; %
\end{tikzpicture}
  \caption{Projektbaum im Unterordner client}
\end{nofloat}  
%\ \\
%\begin{minipage}{1.0\textwidth}
  
Der Ordner ,,build'' enthält die durch CMake generierten VisualStudio
Projekte und fertig gebauten ausführbaren Dateien und DLLs. Die Ordner
,,dlls'' und ,,lib'' enthalten zur Erzeugung und Ausführung der
Projekte nötige externe *dll und *lib Dateien. In den Ordnern
,,include'' und ,,src'' sind schließlich die Quelltexte und Header der
verwendeten Libaries sowie unseres Clients ,,p3dxSteuerung''
enthalten. \\\\
Die Notwendigkeit einer gesondereten Buildinfrastruktur war beim
Server (PioneerMRServer) nicht 
gegeben, da dieser nicht von den Institutsbibliotheken
abhängt. Entsprechend reichte es dort, ein normales VisualStudio Projekt
zu erstellen. 

%\end{minipage}

%%% Local Variables: 
%%% mode: latex
%%% TeX-master: "template"
%%% End: 

\section{Integration bestehender Projekte in ein neues Buildsystem für den
Client}
\label{cha:integr-best-proj}
\subsection{Warum ein neues Buildsystem?}
Die Ballerkennung und der Sonarpartikelfilter hängen von vielen
Altprojekten
ab. Das Beispiel für die Steuerung der Pioneer 3-DX hing zudem von der ARIA
Bibliothek\footnote{http://robots.mobilerobots.com/wiki/ARIA} ab. Bei den
Altprojekten fanden sich auch zum Teil fest kodierte Pfade. Da auch die
Konvertierung der Solutions in das VS 2010 Format meistens fehlschlug und
auch
ARIA nicht mit der mitgelieferten Solution unter VS 2010 gebaut werden
konnte,
haben wir uns für den Client für den Einsatz eines neuen Buildsystems
entschlossen. Aufgrund der Einfacheit der Beschreibung der Buildvorgänge
für
die einzelnen zu integrierenden Projekte und dem Hinzufügen neuer
Unterprojekte, sowie der Möglichkeit, von einem konkreten Buildsystem
abhängig
zu sein sowie existierender Erfahrungen mit dem System, fiel die Wahl auf
CMake\footnote{http://www.cmake.org/}. Bei CMake handelt es sich um ein
open-source Metabuildsystem. Es erzeugt Build-Vorschriften für andere
Buildsysteme (u.a. GNU make, nmake, msbuild). Somit wäre es bei Bedarf auch
Möglich gewesen, auf eine andere Visual Studio Version umzusteigen ohne
die Solutions zu konvertieren oder neu zu erstellen.

\subsection{Integration der Projekte}
Zuerst wurde eine passende Ordnerstrukur angelegt: include für
Header-Dateien,
src für die cpp-Dateien der einzelnen Projekt, bin für eventuell zum Linken
 benötigte
Kompillate, deren Quellen nicht in den Build-Prozess integrierbar waren und
build als Zielordner für erzeugte Buildfiles.

Als nächstes wurde für das Projekt im Ordner des Clients eine
CMakeLists.txt-Datei angelegt:

\begin{lstlisting}[language={},captionpos=b,caption={CMakeLists.txt für das Buildsystem des Clients}]
cmake_minimum_required (VERSION 2.6)
project (TP_WS_2010)

include_directories (include)
include_directories (include/ControlExample)
include_directories (include/erob/)
include_directories (include/Balldetection)
include_directories (include/p3dxSteuerung)
include_directories (include/StochasticLib)
include_directories (include/PioneerMRClient)
include_directories (include/Aria)
include_directories (include/OccupancyGridMap)
include_directories (include/irpVideo/DirectShow)
link_directories (${TP_WS_2010_BINARY_DIR})
link_directories (${TP_WS_2010_SOURCE_DIR}/lib/)
add_subdirectory (src)
\end{lstlisting}

Diese definiert den Namen des Projektes (wichtig
für Referenzen auf z.b. das Stammverzeichnis des Projektes oder das
Buildverzeichnis, da die entprechenden Variablennamen mit dem Projektnamen
+ \_
als Präfix versehen werden. Zudem wurden die include-Verzeichnisse und zum
Linken relevante Verzeichnisse festgelegt. Danach wurde das src-Verzeichnis
 als
Projektunterordner hinzugefügt. Dadurch werden die Anweisungen in der
CMakeLists.txt des Unterordners auch ausgeführt. Dies ermöglicht es, für
die
einzelnen Unterprojekte jeweils eine eigene kleine CMakeLists.txt zu
verwenden.

Die CMakeLists.txt bindet nun alle Sourceordner der Unterprojekte ein:

\begin{lstlisting}[language={},captionpos=b,caption={CMakeLists.txt für den Sourceordner}]
%\begin{lstlisting}
cmake_minimum_required (VERSION 2.6)

add_subdirectory (irpUtils)
add_subdirectory (irpMath)
add_subdirectory (irpImage)
add_subdirectory (irpVideo)
add_subdirectory (irpFeatureExtraction)
add_subdirectory (alglib)
add_subdirectory (DistanceMap)
add_subdirectory (irpCamera)
add_subdirectory (irpV3d)
add_subdirectory (StochasticLib)
add_subdirectory (Visualization)
add_subdirectory (CamCalib)
add_subdirectory (Balldetection)
add_subdirectory (mathtest)
add_subdirectory (SonarParticleFilter)
add_subdirectory (PioneerMRClient)
add_subdirectory (OccupancyGridMap)
add_subdirectory (Balldetection_Test)
add_subdirectory (Aria)
add_subdirectory (p3dxSteuerung)
add_subdirectory (p3dxSteuerung_threaded)
add_subdirectory (calibrate)
#add_subdirectory (ControlExample)
add_subdirectory (utils)
add_subdirectory (kinect)
add_subdirectory (networkTest)
\end{lstlisting}

Bei den Unterprojekten gibt es zwei Arten: Bibliotheken und Anwendungen.

Die CMakeList.txt einer Bibliothek sieht wie folgt aus:
\begin{lstlisting}[language={},captionpos=b,caption={CMakeLists.txt einer Bibliothek am
    Beispiel irpCamera}]
%\begin{lstlisting}
cmake_minimum_required (VERSION 2.6)

file(GLOB src "*.cpp")
file(GLOB includes "../../include/irpCamera/*.h")

add_library (irpCamera ${src} ${includes})
\end{lstlisting}

Die CMakeList.txt einer Anwendung sieht so aus:
\begin{lstlisting}[language={},captionpos=b,caption={CMakeLists.txt des Clients}]
cmake_minimum_required (VERSION 2.6)

file(GLOB src "*.cpp")
file(GLOB includes "../../include/p3dxSteuerung/*.h")
add_executable (p3dxSteuerung_threaded ${src} ${includes})
target_link_libraries (p3dxSteuerung_threaded irpUtils MINPACK fftw rfftw
DirectShow irpFeatureExtraction irpMath  alglib irpImage irpCamera irpVideo
 irpV3d glew CamCalib Visualization StochasticLib OccupancyGridMap
SonarParticleFilter ws2_32 winmm advapi32 Aria DistanceMap Balldetection)

\end{lstlisting}

Zuerst werden alle cpp-Dateien zur Variablen src hinzugfügt, danach die
Includes zur Variablen include. Je nachdem, ob es sich um eine Bibliothek
oder
Anwendung handelt, wird diese mit add\_library(name dateien) oder
add\_executable(name dateien) hinzugefügt. Die Include-Dateien müssen dabei
aber eigentlich nicht explizit hinzugefügt werden, dies geschieht nur,
damit
sie auch in einer VS Solution in der Dateiliste auftauchen. Bei Anwendungen
können nun mit target\_link\_libraries(exename libraries) noch die zu
linkenden
Bibliotheken angegeben werden.

Es gab ein paar Probleme bei der Integration: zum einen mussten zuerst die
Abhängigkeiten zwischen den Unterprojekten ermittelt werden, dies geschah
mit
Hilfe der originalen VS Solutions sowie durch Ausprobieren. Zum anderen gab
 es
manchmal Dateien, die zwar in den Quellen lagen, jedoch in nicht in den
Solutions auftauchten. Diese konnten den Buildprozess stören und mussten
daher
entfernt werden TODO: Beispiel raussuchen, wenn noch überhaupt möglich.
Außerdem waren an ein paar Stellen kleinere Änderungen am Quellcode nötig,
um
ihn mit MSVC 2010 oder der Struktur unseres include-Ordners kompatibel zu
machen.

\subsection{Hinzufügen eines neuen Unterprojektes}
\label{sec:hinz-eines-neuen-unterprojektes}
Das Hinzufügen eines neuen Unterprojektes ist relativ einfach.
Zuerst wird ein Unterverzeichnis für das Projekt im include-Ordner und
eines im
src-Ordner angelegt. Dann kopiert man die includes in den neuen Unterordner
 im
include-Ordner. Dabei ist darauf zu achten, dass die Pfade in
\#include-Direktiven relative zum include-Ordner sein sollten. Danach
kopiert
man die cpp-Dateien in den neuen Unterordner im src-Verzeichnis.

Danach erstellt man in diesem Ordner eine CMakeLists.txt ähnlich wie oben,
jenachdem ob ex sich um eine Anwendung oder Bibliothek handelt. Man fügt
zuerst
die cpp-Dateien und Includes hinzu, fügt dann mit add\_library oder
add\_executable Buildtargets hinzu und gibt eventuell noch zu linkende
Bibliotheken bei Anwendungen an mit target\_link\_libraries.

%%% Local Variables: 
%%% mode: latex
%%% TeX-master: "template"
%%% End: 



\section{Erstellung von Client und Server mit Visual Studio  2010}
Da der Client mit Hilfe des im vorherigenen Abschnitts auf CMake
basierenden Buildystems erzeugt wird, sind hier mehr Schritte als beim
Server notwendig:
\begin{enumerate}
	\item CMake für Windows (Version min. 2.6) herunterladen und installieren
  \footnote{http://www.cmake.org/}
	\item CMake GUI starten
	\item Unter "Where is the source code" den Pfad des Client-Verzeichnisses
 eintragen und unter "Where to build the binaries" den Pfad zu dem
Verzeichnis, wo die Solution für den Client erstellt werden soll
	\item Auf "Configure" klicke, "Visual Studio 2010" und "Use default
native compilers" auswählen, dann auf Finish klicken. Auf das Ende des
Einrichtens warten.
	\item Auf "Generate" klicken
	\item Im ausgewählten Zielverzeichnis befindet sich nun eine
          Visual Studio 2010
Solution (TP\_WS\_2010.sln). 
\item Außerdem befindet sich im Unterordner ,,PioneerMRServer'' im
Projektbaum eine Solution für den
Server (PioneerMRServer.sln). Beide Solutions können ganz normal mit 
Visual Studio 2010 geöffnet und gebaut werden (F7 oder STRG-Alt-F7).
\end{enumerate}

%%% Local Variables: 
%%% mode: latex
%%% TeX-master: "template"
%%% End: 
\chapter{Nutzung der Software}
\label{cha:nutzung}
\section{Client}
\label{sec:client-1}

\subsection{Kalibrierung der Balldetection}
\label{sec:kalibr-der-balld}
Die Ballerkennung muss zunächst kalibriert werden, da ansonsten keine
brauchbaren Ergebnisse erzielt werden können. Dies ist deshalb nötig,
da die Ausrichtung der auf dem Roboter montierten Webcam
entscheidenden Einfluss auf die von der Ballerkennung ermittelten
Werte hat. Zur VisualStudio Solution ,,TP2010/11'' gehört auch ein
Unterprojekt calib. Nach Erstellung der ausführbaren Datei calib.exe
ist diese an folgende Stelle des Projektbaums zu kopieren:
\verb|\client\src\calibrate\| \\ 
Für die folgenden Schritte werden dann ein Schachbrett sowie ein
Maßband oder Ähnliches benötigt. 
Danach geht man folgendermaßen vor:
\begin{itemize}
\item Man stellt das Schachbrett und den Roboter in einen Abstand von
  ca 1 Meter voneinander entfernt so auf, dass die Kamera auf die
  Mitte des Schachbrettes zeigt. 
% TODO: bild einfügen
\item Danach misst man mit dem Maßband die Entfernung zwischen Roboter
  und Schachbrett, sowie den Abstand vom Schachbrett zum Boden (im
  Normallfall 25 cm). Die
  beiden Enden der Entfernungstrecke zeigt folgendes Bild:
 \begin{nofloat}{figure}\centering
    \includegraphics[width=0.55\linewidth]{bilder/camToGround_red}
    \caption{Abstand zwischen Boden und Schachbrett}
  \end{nofloat}
%TODO: bild einfügen
\item Anschließend startet man die Batchdatei \verb|calibrate.cmd| im
  Ordner \verb|\client\src\calibrate\|. Die Batchdatei fragt nun die
  Kameranummer, sowie die im vorherigen Schritt ermittelten 
 Entfernungen in mm ab und startet anschließend durch Drücken der
 Enter-Taste die Erkennung:
 \begin{nofloat}{figure}\centering
    \includegraphics[width=1\linewidth]{bilder/calibrate1}
    \caption{Abfrage der zur Kalibrierung nötigen Parameter}
  \end{nofloat}\newpage
\item Die Kalibrierung ist erfolgreich, wenn folgendes Muster im
  Kamerabild angezeigt wird:
  \begin{nofloat}{figure}\centering
    \includegraphics[width=0.7\linewidth]{bilder/calibrate2}
    \caption{Darstellung einer erfolgreichen Kalibrierung der Ballerkennung}
  \end{nofloat}
\item Dann ist im Batch-Skript die Ausführung
  des Kalibiertools mit \textless STRG \textgreater -C zu unterbrechen. Auf die Nachfrage,
  ob auch die Batchdatei unterbrochen werden soll, gibt man ,,n'' ein:
 \begin{nofloat}{figure}\centering
    \includegraphics[width=\linewidth]{bilder/calibrate3}
  \end{nofloat}\newpage
\item Danach sollte ein neuer Ordner im aktuellen Verzeichnis mit den
  Kalibrierdaten sein. Er hat als Bezeichnung die Kameranummer
  erhalten. Im folgenden Bild sieht man das Ergebnis einer
  erfolgreichen Kalibrierung für die Kamera mit der Nummer 4, es wurde
  ein neues Verzeichnis \verb|cam4| mit den Kalibrierungsdateien erzeugt:
 \begin{nofloat}{figure}\centering
    \includegraphics[width=\linewidth]{bilder/calibrate4_red}
  \end{nofloat}
\end{itemize}
%%% Local Variables: 
%%% mode: latex
%%% TeX-master: "template"
%%% End: 


\subsection{Einrichtung des Clients}%Konfiguration und Nutzung des Clients}
\label{sec:einrichtung-client}
Zunächst müssen die Webcam des Roboters und dessen serielle
Schnittstelle  mit dem Laptop, auf dem die Clientsoftware läuft,
verbunden  werden. Anschließend wird der Client über  die Datei \verb|config.cfg| eingerichtet. Eine
Erklärung und ein Beispiel finden sich im Abschnitt
\ref{sec:funktionsweise} auf Seite \pageref{aufbau_config}. Die Datei
muss sich im gleichen Verzeichnis wie die ausführbare Datei
\verb|p3dxSteuerung_threaded.exe| befinden. Ausserdem benötigt der
Client ein Batchskript zum Starten, die
Daten der Kalibrierung und Karten für die Lokalisierung. Idealerweise
erzeugt man sich für die *exe-Datei und anderen benötigten Dateien ein
eigenes Verzeichnis, wo alle benötigten Dateien rein kopiert werden.  Hierzu
empfiehlt sich der in Abbildung \ref{aufbau_laufzeit} skizzierte
Aufbau des Laufzeitverzeichnisses. Die Bedeutung der Elemente der
Verzeichnisstruktur ist der Tabelle
\ref{bedeutung_namen} zu entnehmen.
\begin{nofloat}{figure}%{l}{1\textwidth}
\label{aufbau_laufzeit}
\centering
\begin{tikzpicture}
%\tikz 
[font=\footnotesize,
       grow=right, level 1/.style={sibling distance=5em}
                   level 2/.style={sibling distance=6em}, level distance=5cm]
  \node {p3dxSteuerung} % root
     child { node {config}}
     child {node {client}
       child {node {Aria.dll}}
       child {node {calib.exe}}
       child {node {Calibrate.cmd}}
       child {node {rayCasting}
         child{node {Elevator2cm.bmp}}
         child{node {Elevator.BMP}}
         child{node {OccuMap.bmp}}
       }
       child {node {config.cfg}}
       child {node{balldetection\_test.exe}}
       child {node {p3dxSteuerung\_threaded.exe}}
       child {node {libzmq.dll}}
      % child {node {CamNo}}
       child {node {Start.bat}}
       child {node {wskKarte}
         child{node{Box.BMP}}
         child{node{Box.txt}}
         child{node {Elevator2.txt}}
         child{node {Elevator.BMP}}
         child{node {Elevator.txt}}
         }
       child {node {wtee.exe}}
   };
\end{tikzpicture}
\caption{Empfohlener Aufbau des Laufzeitverzeichnisses}  
\end{nofloat}
\begin{nofloat}{table}{
    %\begin{table}
      \centering
      \begin{tabular}{|c|p{0.8\linewidth}|}
        \hline 
        Name & Bedeutung \\ \hline
        config & Vereichnis mit den *xml-Kalibrierungsdateien\\ \hline
        client & Verzeichnis mit den Programmdateien des Clients\\ \hline
        Aria.dll, libzmq.dll & DLLs, die vom Client zur Ansteuerung
        des Roboters und zur Interprozesskommunikation benötigt werden
        \\ \hline
        %Datei der libAria, wird zur Ansteuerung des
        %Roboters benötigt\\ \hline
        calib.exe & Hilfstool zur Kalibrierung der Ballerkennung
        \footnote{Basierend auf der balldetection\_test.exe von Tobias Breuer}\\ \hline
        Calibrate.cmd &  Wrapper-Batchskript für die Kalibrierung \\ \hline
        rayCasting,  wskKarte & Verzeichnisse mit Raumkarten im
        BMP-Format\\ \hline
        config.cfg & Konfigurationsdatei für den Client\\ \hline
        balldetection\_test.exe & Miniprogramm zum Testen der
        Ballerkennung \footnote{Geschrieben von Tobias Breuer}\\ \hline
        wtree.exe & Windows-Tool, um Bildschirmausgaben von
        DOS-Programmen zu loggen und trotzdem am Bildschirm
        auszugeben.\footnote{OpenSource von Ryan Buhl unter der
          Mozilla Public License}\\ \hline
        Start.bat & Batch-Skript, um den Client auszuführen und
        gleichzeitig ein Logging mittels wtree.exe zu ermöglichen.
        %und gleichzeitigen
        %Ausgaben voder Ausgabe des Clients
        %\footnote{Dieses T
       \\ \hline
      \end{tabular}
      \caption{Die Dateien des Laufzeitverzeichnisses und ihre Bedeutung}
      \label{bedeutung_namen}
   % \end{table}
}
\end{nofloat}
%%% Local Variables: 
%%% mode: latex
%%% TeX-master: "template"
%%% End: 


\subsection{Benutzung des Clients}
\label{sec:benutz-des-clients}
Nach Konfiguration des Clients muss zunächst der Server gestartet
werden, da der Client eine funktionierende Verbindung zum Server
voraussetzt. Ist keine vorhanden, bricht er die Programmausführung ab.
Nach Start des Servers  positioniert man den Roboter in der Mitte des
Raumes, in dem die Ballsuche stattfinden soll. Anschließend schaltet
man den Roboter ein und startet den Client mit der
Stapelverarbeitungsdatei \verb|Start.bat|. Der Client startet nun und
nimm zunächst soviele Rotationen um die eigene Ache vor, wie in der
\verb|config.cfg| festgelegt worden sind. Diese dienen dazu, den
Sonar-Partikelfilter zu initialisieren, sodass der Client dann die
Position des Roboters im Raum bestimmen kann. Nach erfolgreicher
Initialisierung verbindet sich der Client mit den Server und wartet,
bis dieser die Suche startet. Sobald vom Server die Suche gestartet
wurde, wird der Client vom Server eine Liste von Punkten anfordern,
die er dann abfahren wird. Gleichzeitig öffnet sich ein Fenster, dass
das aktuelle Bild der Kamera zeigt. Es wird während der gesamten Suche
ständig aktualisiert. \\\\
Ab diesen Zeitpunkt muss man als Benutzer
nur noch den Ablauf überwachen, falls es unerwarteterweise zu
Kollisionen mit der Wand oder anderen Robotern kommt. Dies sollte zwar
aufgrund der vom Server vorgenommen Kollisionerkennung nicht
passieren, allerdings lassen sich ,,false Positives'' nicht generell
ausschließen, mehr dazu im nächsten Kapitel. Erkennt der Client
schließlich einen Ball im aktuellen Kamerabild, wird der Ball im Bild
markiert:
\begin{nofloat}{figure}\centering
\includegraphics[width=0.75\linewidth]{bilder/balldetect}
\caption{Kamerabild bei erfolgreicher Ballerkennung}  
\end{nofloat}

Anschließend wird aus der mit den Sonar-Partikelfilter ermittelten
Position des Roboters im Raum und von der Ballerkennung zurückgebenen
Entfernung und Winkel vom Ball zum Roboter die Position des Balls im
Raum bestimmt und an den Server übermittelt. Dieser weiß nun, dass die
Ballsuche erfolgreich beendet ist und schickt allen verbundenen
Clients eine Aufforderung die Suche zu beenden. In der Folge stoppen
die Clients die Suche.
%Zunächst 
%%% Local Variables: 
%%% mode: latex
%%% TeX-master: "template"
%%% End: 

%\section{Integration bestehender Projekte in ein neues Buildsystem für den
Client}
\label{cha:integr-best-proj}
\subsection{Warum ein neues Buildsystem?}
Die Ballerkennung und der Sonarpartikelfilter hängen von vielen
Altprojekten
ab. Das Beispiel für die Steuerung der Pioneer 3-DX hing zudem von der ARIA
Bibliothek\footnote{http://robots.mobilerobots.com/wiki/ARIA} ab. Bei den
Altprojekten fanden sich auch zum Teil fest kodierte Pfade. Da auch die
Konvertierung der Solutions in das VS 2010 Format meistens fehlschlug und
auch
ARIA nicht mit der mitgelieferten Solution unter VS 2010 gebaut werden
konnte,
haben wir uns für den Client für den Einsatz eines neuen Buildsystems
entschlossen. Aufgrund der Einfacheit der Beschreibung der Buildvorgänge
für
die einzelnen zu integrierenden Projekte und dem Hinzufügen neuer
Unterprojekte, sowie der Möglichkeit, von einem konkreten Buildsystem
abhängig
zu sein sowie existierender Erfahrungen mit dem System, fiel die Wahl auf
CMake\footnote{http://www.cmake.org/}. Bei CMake handelt es sich um ein
open-source Metabuildsystem. Es erzeugt Build-Vorschriften für andere
Buildsysteme (u.a. GNU make, nmake, msbuild). Somit wäre es bei Bedarf auch
Möglich gewesen, auf eine andere Visual Studio Version umzusteigen ohne
die Solutions zu konvertieren oder neu zu erstellen.

\subsection{Integration der Projekte}
Zuerst wurde eine passende Ordnerstrukur angelegt: include für
Header-Dateien,
src für die cpp-Dateien der einzelnen Projekt, bin für eventuell zum Linken
 benötigte
Kompillate, deren Quellen nicht in den Build-Prozess integrierbar waren und
build als Zielordner für erzeugte Buildfiles.

Als nächstes wurde für das Projekt im Ordner des Clients eine
CMakeLists.txt-Datei angelegt:

\begin{lstlisting}[language={},captionpos=b,caption={CMakeLists.txt für das Buildsystem des Clients}]
cmake_minimum_required (VERSION 2.6)
project (TP_WS_2010)

include_directories (include)
include_directories (include/ControlExample)
include_directories (include/erob/)
include_directories (include/Balldetection)
include_directories (include/p3dxSteuerung)
include_directories (include/StochasticLib)
include_directories (include/PioneerMRClient)
include_directories (include/Aria)
include_directories (include/OccupancyGridMap)
include_directories (include/irpVideo/DirectShow)
link_directories (${TP_WS_2010_BINARY_DIR})
link_directories (${TP_WS_2010_SOURCE_DIR}/lib/)
add_subdirectory (src)
\end{lstlisting}

Diese definiert den Namen des Projektes (wichtig
für Referenzen auf z.b. das Stammverzeichnis des Projektes oder das
Buildverzeichnis, da die entprechenden Variablennamen mit dem Projektnamen
+ \_
als Präfix versehen werden. Zudem wurden die include-Verzeichnisse und zum
Linken relevante Verzeichnisse festgelegt. Danach wurde das src-Verzeichnis
 als
Projektunterordner hinzugefügt. Dadurch werden die Anweisungen in der
CMakeLists.txt des Unterordners auch ausgeführt. Dies ermöglicht es, für
die
einzelnen Unterprojekte jeweils eine eigene kleine CMakeLists.txt zu
verwenden.

Die CMakeLists.txt bindet nun alle Sourceordner der Unterprojekte ein:

\begin{lstlisting}[language={},captionpos=b,caption={CMakeLists.txt für den Sourceordner}]
%\begin{lstlisting}
cmake_minimum_required (VERSION 2.6)

add_subdirectory (irpUtils)
add_subdirectory (irpMath)
add_subdirectory (irpImage)
add_subdirectory (irpVideo)
add_subdirectory (irpFeatureExtraction)
add_subdirectory (alglib)
add_subdirectory (DistanceMap)
add_subdirectory (irpCamera)
add_subdirectory (irpV3d)
add_subdirectory (StochasticLib)
add_subdirectory (Visualization)
add_subdirectory (CamCalib)
add_subdirectory (Balldetection)
add_subdirectory (mathtest)
add_subdirectory (SonarParticleFilter)
add_subdirectory (PioneerMRClient)
add_subdirectory (OccupancyGridMap)
add_subdirectory (Balldetection_Test)
add_subdirectory (Aria)
add_subdirectory (p3dxSteuerung)
add_subdirectory (p3dxSteuerung_threaded)
add_subdirectory (calibrate)
#add_subdirectory (ControlExample)
add_subdirectory (utils)
add_subdirectory (kinect)
add_subdirectory (networkTest)
\end{lstlisting}

Bei den Unterprojekten gibt es zwei Arten: Bibliotheken und Anwendungen.

Die CMakeList.txt einer Bibliothek sieht wie folgt aus:
\begin{lstlisting}[language={},captionpos=b,caption={CMakeLists.txt einer Bibliothek am
    Beispiel irpCamera}]
%\begin{lstlisting}
cmake_minimum_required (VERSION 2.6)

file(GLOB src "*.cpp")
file(GLOB includes "../../include/irpCamera/*.h")

add_library (irpCamera ${src} ${includes})
\end{lstlisting}

Die CMakeList.txt einer Anwendung sieht so aus:
\begin{lstlisting}[language={},captionpos=b,caption={CMakeLists.txt des Clients}]
cmake_minimum_required (VERSION 2.6)

file(GLOB src "*.cpp")
file(GLOB includes "../../include/p3dxSteuerung/*.h")
add_executable (p3dxSteuerung_threaded ${src} ${includes})
target_link_libraries (p3dxSteuerung_threaded irpUtils MINPACK fftw rfftw
DirectShow irpFeatureExtraction irpMath  alglib irpImage irpCamera irpVideo
 irpV3d glew CamCalib Visualization StochasticLib OccupancyGridMap
SonarParticleFilter ws2_32 winmm advapi32 Aria DistanceMap Balldetection)

\end{lstlisting}

Zuerst werden alle cpp-Dateien zur Variablen src hinzugfügt, danach die
Includes zur Variablen include. Je nachdem, ob es sich um eine Bibliothek
oder
Anwendung handelt, wird diese mit add\_library(name dateien) oder
add\_executable(name dateien) hinzugefügt. Die Include-Dateien müssen dabei
aber eigentlich nicht explizit hinzugefügt werden, dies geschieht nur,
damit
sie auch in einer VS Solution in der Dateiliste auftauchen. Bei Anwendungen
können nun mit target\_link\_libraries(exename libraries) noch die zu
linkenden
Bibliotheken angegeben werden.

Es gab ein paar Probleme bei der Integration: zum einen mussten zuerst die
Abhängigkeiten zwischen den Unterprojekten ermittelt werden, dies geschah
mit
Hilfe der originalen VS Solutions sowie durch Ausprobieren. Zum anderen gab
 es
manchmal Dateien, die zwar in den Quellen lagen, jedoch in nicht in den
Solutions auftauchten. Diese konnten den Buildprozess stören und mussten
daher
entfernt werden TODO: Beispiel raussuchen, wenn noch überhaupt möglich.
Außerdem waren an ein paar Stellen kleinere Änderungen am Quellcode nötig,
um
ihn mit MSVC 2010 oder der Struktur unseres include-Ordners kompatibel zu
machen.

\subsection{Hinzufügen eines neuen Unterprojektes}
\label{sec:hinz-eines-neuen-unterprojektes}
Das Hinzufügen eines neuen Unterprojektes ist relativ einfach.
Zuerst wird ein Unterverzeichnis für das Projekt im include-Ordner und
eines im
src-Ordner angelegt. Dann kopiert man die includes in den neuen Unterordner
 im
include-Ordner. Dabei ist darauf zu achten, dass die Pfade in
\#include-Direktiven relative zum include-Ordner sein sollten. Danach
kopiert
man die cpp-Dateien in den neuen Unterordner im src-Verzeichnis.

Danach erstellt man in diesem Ordner eine CMakeLists.txt ähnlich wie oben,
jenachdem ob ex sich um eine Anwendung oder Bibliothek handelt. Man fügt
zuerst
die cpp-Dateien und Includes hinzu, fügt dann mit add\_library oder
add\_executable Buildtargets hinzu und gibt eventuell noch zu linkende
Bibliotheken bei Anwendungen an mit target\_link\_libraries.

%%% Local Variables: 
%%% mode: latex
%%% TeX-master: "template"
%%% End: 

%\chapter{Der Client}
Der Client in unserer Client-Server-Architektur ist für das Anfahren der
einzelnen Routenpunkte und die Erkennung des Balls zuständig. Dazu haben
wir für die Lokalisierung die uns zu Verfügung gestellte Implementierung
eines Sonar-Partikelfilters und für die Ballerkennung die Implementierung
aus einer Bachelorarbeit von Tobias Breuer verwendet.

Am Partikelfilter haben wir ein paar Änderungen vorgenommen. Es gab keine
Möglichkeit, von außerhalb des Filters auf die Koordinaten und die
Ausrichtung des Roboters zuzugreifen. Dazu wurden die Methoden StartFilter
und findBestParticle so erweitert, dass die aktuelle Position des Roboters
und seine Ausrichtung als Zeiger auf das erste Element eines double-Arrays
([x,y,Theta, Partikelwsk.]) zurückgegeben wird. Da dieses in
findBestParticle dynamisch mit new alloziiert wird, muss es, wenn es nicht
mehr beötigt wird, mit delete[] gelöscht werden. Außerdem ist nun möglich
die Anzahl der Sonare von außerhalb des Filters ohne neukompillieren
einzustellen. Die Visualisierungen wurden aus Performancegründen entweder
entfernt oder deaktiviert. Zudem wurde für eine einfachere Auswertung und
weil die Partikelvisualisierung öfter zu Abstürzen unseres Programms
geführt hat der Filter um eine Funktion erweitert, die, wenn in der
particleSet.cpp takeADump true ist, bei jedem Filtern alle Partikel als
Tripel aus x,y und der Partikelwsk. in eine Datei schreibt. Die Datei hat
dabei als Prefix eine Zahl im Namen, die bei 0 beginnend nach jedem Filtern
 um 1 erhöht wird. Passend dazu wurde ein Visualisierer geschrieben, der
die Partikelmenge grafisch darstellt, auf den noch später eingegangen wird.

Der Client selber arbeitet folgendermaßen:

Beim Start wird die Datei config.cfg eingelesen, wenn diese nicht gefunden
wird, nutzt der Client die fest eingestellten Standardwerte. Die config.cfg
hat folgende Struktur:

\begin{lstlisting}
#Kommentarzeile (wird ignoriert)
Sonaranzahl als int
#Kommentarzeile (wird ignoriert)
COM-Port, an dem der Roboter angeschlossen ist, als String
#Kommentarzeile (wird ignoriert)
Schrittweite fuer die Anlerndrehungen als int in Grad
#Kommentarzeile (wird ignoriert)
Anzahl der 360Grad Drehungen als int
#Kommentarzeile (wird ignoriert)
Server-IP als String
#Kommentarzeile (wird ignoriert)
Pixel in der Karte pro mm als double
\end{lstlisting}

Ein Beispielkonfiguration für den Fahrstuhlvorraum wäre z. B.
\begin{lstlisting}
#Anz Sensoren
8
#COM Port
COM5
#Schrittweite fuer die Anlerndrehungen
15
#Anzahl der 360Grad Drehungen
1
#Server-IP
134.169.36.241
#Pixel pro mm
0.05081
\end{lstlisting}
Die Standardwerte sind (16, COM3, 45, 0, 127.0.0.1, 1).

Danach wird die Verbindung zum Roboter hergestellt und der Partikelfilter
mit 10000 Partikeln initialisiert und das erste Mal gefiltert. Danach
erfolgt die konfigurierte Anzahl an Drehungen in der konfigurierten
Schrittweite um die erste Positionierung zu verbessern. Nach jedem
Drehschritt wird einmal gefiltert. Nach den Drehungen wird noch einmal
gefiltert und der Rückgabewert als Startposition genommen. Erst jetzt wird
die Verbindung zum Server hergestellt und die Roboterposition übermittelt,
sowie ein Thread gestartet, der regelmäßig die RobotInformation,
PathInformation und SearchInformation mit dem Server abgleicht. Der Client
wartet dann erstmal so lange, bis der Server in der SearchInformation pause
 = false sendet. Sobald pause = true, wird ein weiterer Thread zur
Ballerkennung gestartet und der Client geht in seine Fahrschleife über, die
 erst endet, wenn der Ball gefunden wurde oder stoppt, wenn pause = true.
 Der Client wartet dann auf den nächsten anzufahrenden Punkt vom Server,
sobald er diesen erhalten hat fährt der Roboter den Punkt an.

Dies geschieht schrittweise:\\
Zuerst wird die Richtung und der Drehwinkel der Drehung für eine
Ausrichtung zum Punkt berechnet. Dazu wird der Winkel zwischen dem
Ausrichtungsvektor des Roboters und dem Punkt berechnet und dann geprüft,
ob der Vektor zwischen Roboterposition und dem Punkt um den ermittelten
Winkel im Uhrzeigersinn oder gegen den Uhrzeigersinn von der
Roboterausrichtung aus liegt um die Drehrichtung zu ermitteln. Dann wird
der Roboter entprechend der ermittelten Werte zum Punkt hin gedreht.
Dann wird, falls pause in der SearchInformation auf true ist, solange
gewartet, bis pause wieder false ist. Danach wird
drivingTime * driveLengthWeight ms gefahren, wobei drivingTime die
Fahrtzeit pro Schritt ist und driveLengthWeight ein Anpassungsfaktor der
Fahrzeit ist für den Fall, dass die Reststrecke in weniger als drivingTime
zurückgelegt werden kann. Dazu wird aus den gefahrenen Strecken in jedem
Fahrtschritt und der jeweiligen Fahrtzeit die Strecke in px pro ms
berechnet und über die Schritte der exponentiell geglätteter Mittelwert
berechnet, woraus dann driveLengthWeight für die Reststrecke berechnet wird
, wenn die zu klein für die normale drivingTime ist.

Dann wird eine neue Ausrichtung zum Punkt berechnet und mit der aktuellen
 verglichen, verkürzt sich dadurch die Entfernung zum Ziel, wird der
Roboter neu ausgerichtet, ansonsten wird die alte Ausrichtung beibehalten.
Danach wird wieder eine Teilstrecke gefahren usw. bis sich dem Punkt auf
einen Abstand < delta\_pos angenährt wurde.
Danach wird gewartet, bis der Server das erreichen des Punktes registriert
hat und ein neuer, anzufahrender punkt übermittelt wurde.








%%% Local Variables: 
%%% mode: plain-tex
%%% TeX-master: "template"
%%% End: 
%Wie haben wir die Aufgabe gelöst? 
                 %Softwarearchitektur!  
                 % - client: inklusive SonarParticelFilter, Balldetection 
                 % - server: inklusive Bahnplanung und ggf. weiterer 
                 % Algorithmen 
                 % 

\chapter{Evaluation}
\label{cha:evaluation}
In diesen Abschnitt nehmen wir eine Bewertung der einzelnen
Bestandteile der Software vor. Dabei geht es zum einen um die
Client-Server Kommunikation, die Bahnplanung und Kollisionsvermeidung,
und zum Anderen um die Lokalisierung des Clients im Raum mit dem
Partikelfilter und die Ballerkennung.
\section{Evaluierung des Servers}
\label{sec:eval-des-serv}
%\input{lokalisierung}
\input{eva_ball}

%%% Local Variables:
%%% mode: latex
%%% TeX-master: "template"
%%% End:


%Bei der Kommunikation war zu testen, pob
%\input{eva_kommunikation}
\subsection{Client-Server Kommunikation}
Die zu testende Funktionalität ist das Übermitteln der Zielpunkte der
geplanten Bahn an den Client und der Empfang der Roboterpositionen vom
Client durch den Server. Bei unseren Experimenten traten keinerlei
Probleme auf. 
\subsection{Bahnplanung}
Die Bahnplanung soll ja eine möglichst zuvällige Auswahl von Punkten
treffen, sodass die Roboter möglichst viele im Raum verteile
Positionen anfahren, sodass irgendwann der Roboter den Ball nahe genug
kommt, um ihm zu erkennen. Dies leistet die Bahnplanung ohne Probleme.
\subsection{Kollisonsvermeidung}
Hier ergab sich das Problem, dass wir bei unseren Versuchen nur mit
einen Roboter arbeiten konnten, da unsere Laptops schon durch den
ersten Roboter, sowie den Server ,,belegt'' waren. Somit konnten wir
mit den realen Roboter nur testen, ob die Kollision mit Wänden
vermieden wurde. Dies klappte in der Tat ohne Probleme. Nur einmal
wurde die Wand doch angefahren. Dies war darauf zurückzuführen, dass
die Positionserkennung mit dem Sonar-Partikelfilter kurzfristig einen
Ausreißer hatte (vgl. hierzu auch den Abschnitt
\ref{sec:lokal-mit-den} ab Seite \pageref{sec:lokal-mit-den}). Somit
war aus Sicht der Kollisonsvermeidung keine Gefahr vorhanden und
brauchte somit auch nicht abgewehrt werden. Im Endeffekt ist dieser
,,Fehlschlag'' also kein Zeichen, dass die Kollisionsvermeidung nicht
funktionieren würde. \\\\
Um nun das Verhalten der Bahnplanung und Kollisionsvermeidung bei
mehreren Robotern zu testen, wurde auf den im Abschnitt
\ref{serv:testclient} auf Seite \pageref{serv:testclient}
beschriebenen Testclient zurückgegriffen, der das Verhalten des
Clients emuliert. Wir haben insgesamt drei virtuelle Testclients mit
dem Server verbunden und konnten somit die Kollisionsvermeidung testen.
%Bei der Bahnplanung ergab sich das Problem, dass uns nur ein Roboter
%zum Testen zur Verfügung 
%input{eva_bahnplanung}
%\input{eva_kollision}
%Client-Server Kommunikation, die Bahnplanung und Kollisionsvermei
%%% Local Variables:
%%% mode: latex
%%% TeX-master: "template"
%%% End:

\section{Lokalisierung mit den Sonar-Partikelfilter}
Bei unseren Versuchen die Position des Roboters im Raum zu bestimmen,
stiessen wir recht früh auf Probleme: So wurde regelmäßig eine
komplett falsche Position im Raum bestimmt oder aber (gerade wenn der
Roboter sich nahe einer Wand befand) die dem Roboter gegenüberliegende
Position am anderen Ende des Raums als Position erkannt.  Nach
mehreren Versuchen stellten wir schließlich fest, dass die
Lokalisierung besonders zuverlässig war, wenn der Roboter beim Start
des Clients sich in der Mitte des Raums befand, sodass bei der
Initialisierung des Partikelfilters durch die initiale Rotation der
Abstand zu den Wänden zu beiden Seiten gleich war. Um nun der Ursache
dieses Phänomens auf die Spur zu kommen, haben wir dann unser eigenes
Visualisierungsskript geschrieben. Wir beschreiben nun zunächst die
Funktionsweise und Installation des Skriptes, bevor wir uns den
Ergebnissen zuwenden.
\input{visualisierung}

%\input{visualisierung}

%%% Local Variables:
%%% mode: latex
%%% TeX-master: "template"
%%% End:


\subsection{False-Positives bei der Ballerkennung}
\label{sec:false-positives-bei}
Wie bereits im Abschnitt \ref{sec:balldetection} erwähnt, kann es bei
der Ballerkennung zu False-Positives kommen. Diese liegen in der
Funktionsweise der Ballerkennung begründet: Nach Aufnahme des
Kamerabildes wird das Bild vom RGB- in den HSV-Farbraum umgewandelt,
der anschließend nach roten Partikeln gefiltert wird. Erreichen diese
einen gewissen Schwellwert, werden diese nach kreisförmigen Objekten
durchsucht. Anschließend wird die Wahrscheinlichkeit berechnet, ob es
sich dabei um einen Ball handelt. Wenn nun aber im Kamerabild so ein
rotes Objekt ist (z.B: Ein anderer p3dx-Roboter), wird dieses mit
hoher Wahrscheinlichkeit als ,,Ball'' erkannt. In unseren Versuchen
wurden unter anderen der Feuerlöscher des Labors, aber auch die roten
Schränke gerne als ,,Bälle'' erkannt. Dies konnten wir durch Aufbau
eines geeigneten Sichtschutzes verhindern, allerdings kam es immer
noch zu False-Positives an Stellen, wo weder ein rotes Objekt, noch
sonst irgendein Objekt vorhanden war. Wir vermuten , dass dann höchstwahrscheinlich im
Licht der den Vorraum des Fahrstuhl vor dem Robotiklabor beleuchtenden
Lampen soviele Rot-Partikel vorhanden sind, dass auch da unter
Umständen False-Positives auftauchen können.
%%% Local Variables:
%%% mode: latex
%%% TeX-master: "template"
%%% End:


%%% Local Variables:
%%% mode: latex
%%% TeX-master: "template"
%%% End:

%\section{Lokalisierung mit den Sonar-Partikelfilter}
Bei unseren Versuchen die Position des Roboters im Raum zu bestimmen,
stiessen wir recht früh auf Probleme: So wurde regelmäßig eine
komplett falsche Position im Raum bestimmt oder aber (gerade wenn der
Roboter sich nahe einer Wand befand) die dem Roboter gegenüberliegende
Position am anderen Ende des Raums als Position erkannt.  Nach
mehreren Versuchen stellten wir schließlich fest, dass die
Lokalisierung besonders zuverlässig war, wenn der Roboter beim Start
des Clients sich in der Mitte des Raums befand, sodass bei der
Initialisierung des Partikelfilters durch die initiale Rotation der
Abstand zu den Wänden zu beiden Seiten gleich war. Um nun der Ursache
dieses Phänomens auf die Spur zu kommen, haben wir dann unser eigenes
Visualisierungsskript geschrieben. Wir beschreiben nun zunächst die
Funktionsweise und Installation des Skriptes, bevor wir uns den
Ergebnissen zuwenden.
\section{Visualisierung der Partikelmengen des Partikelfilters}
Da die im Partikelfilter integrierte Visualisierung der Partikelmenge
nicht immer funktionierte und es auch keine direkte SPeichermöglichkeit der
 Bilder gab, haben wir den Partikelfilter, so erweitert, dass die
Partikelmenge in eine Datei geschrieben werden kann (
\ref{sec:sonarparticlefilter}) und passend dazu ein Skript (in Python unter
 Benutzung von pygame geschrieben), welches die Partikel anhand der Daten
 aus einer Partikelmengendatei über ein Bild legt. Dieses ist im Ordner
 Visualisierung im SVN-Repository unter dem Namen visualize.py zu finden.

 \subsection{HOWTO: Installieren der Abhängigkeiten des
Visualisierungsskriptes und Benutzung dessen}
 \begin{itemize}
	 \item Python 3.2 herunterladen und installieren\footnote{http://python.org/ftp/python/3.2.2/python-3.2.2.msi}
	 \item pygame 1.9.2a0 für Python 3.2 installieren \footnote{http://pygame.org/ftp/pygame-1.9.2a0.win32-py3.2.msi}
	 \item Python 3.2 zum PATH hinzufügen
	 \item cmd/Eingabeaufforderung öffnen
	 \item In den Ordner Visualisierung des Repositories wechseln
	 \item Das Visualisierungsskript kann nun folgendermaßen benutzt werden:\\
	 		\lstinline|python visualize.py (Dateiname des Bildes) (Dateiname der Partikelmengendatei)| \\
 			z.b. \lstinline|python visualize.py OccuMap.bmp visual.log|
	 \item Das Ausgabebild hat dann den Namen V\_(Dateiname der Partikelmengendatei)\_(Dateiname des Bildes)
\end{itemize}

%\section{Visualisierung der Partikelmengen des Partikelfilters}
Da die im Partikelfilter integrierte Visualisierung der Partikelmenge
nicht immer funktionierte und es auch keine direkte SPeichermöglichkeit der
 Bilder gab, haben wir den Partikelfilter, so erweitert, dass die
Partikelmenge in eine Datei geschrieben werden kann (
\ref{sec:sonarparticlefilter}) und passend dazu ein Skript (in Python unter
 Benutzung von pygame geschrieben), welches die Partikel anhand der Daten
 aus einer Partikelmengendatei über ein Bild legt. Dieses ist im Ordner
 Visualisierung im SVN-Repository unter dem Namen visualize.py zu finden.

 \subsection{HOWTO: Installieren der Abhängigkeiten des
Visualisierungsskriptes und Benutzung dessen}
 \begin{itemize}
	 \item Python 3.2 herunterladen und installieren\footnote{http://python.org/ftp/python/3.2.2/python-3.2.2.msi}
	 \item pygame 1.9.2a0 für Python 3.2 installieren \footnote{http://pygame.org/ftp/pygame-1.9.2a0.win32-py3.2.msi}
	 \item Python 3.2 zum PATH hinzufügen
	 \item cmd/Eingabeaufforderung öffnen
	 \item In den Ordner Visualisierung des Repositories wechseln
	 \item Das Visualisierungsskript kann nun folgendermaßen benutzt werden:\\
	 		\lstinline|python visualize.py (Dateiname des Bildes) (Dateiname der Partikelmengendatei)| \\
 			z.b. \lstinline|python visualize.py OccuMap.bmp visual.log|
	 \item Das Ausgabebild hat dann den Namen V\_(Dateiname der Partikelmengendatei)\_(Dateiname des Bildes)
\end{itemize}

%%% Local Variables:
%%% mode: latex
%%% TeX-master: "template"
%%% End:

%\section{Visualisierung der Partikelmengen des Partikelfilters}
Da die im Partikelfilter integrierte Visualisierung der Partikelmenge
nicht immer funktionierte und es auch keine direkte SPeichermöglichkeit der
 Bilder gab, haben wir den Partikelfilter, so erweitert, dass die
Partikelmenge in eine Datei geschrieben werden kann (
\ref{sec:sonarparticlefilter}) und passend dazu ein Skript (in Python unter
 Benutzung von pygame geschrieben), welches die Partikel anhand der Daten
 aus einer Partikelmengendatei über ein Bild legt. Dieses ist im Ordner
 Visualisierung im SVN-Repository unter dem Namen visualize.py zu finden.

 \subsection{HOWTO: Installieren der Abhängigkeiten des
Visualisierungsskriptes und Benutzung dessen}
 \begin{itemize}
	 \item Python 3.2 herunterladen und installieren\footnote{http://python.org/ftp/python/3.2.2/python-3.2.2.msi}
	 \item pygame 1.9.2a0 für Python 3.2 installieren \footnote{http://pygame.org/ftp/pygame-1.9.2a0.win32-py3.2.msi}
	 \item Python 3.2 zum PATH hinzufügen
	 \item cmd/Eingabeaufforderung öffnen
	 \item In den Ordner Visualisierung des Repositories wechseln
	 \item Das Visualisierungsskript kann nun folgendermaßen benutzt werden:\\
	 		\lstinline|python visualize.py (Dateiname des Bildes) (Dateiname der Partikelmengendatei)| \\
 			z.b. \lstinline|python visualize.py OccuMap.bmp visual.log|
	 \item Das Ausgabebild hat dann den Namen V\_(Dateiname der Partikelmengendatei)\_(Dateiname des Bildes)
\end{itemize}


%%% Local Variables:
%%% mode: latex
%%% TeX-master: "template"
%%% End:
%Beschreibung einiger Versuche (sowohl erfolgreich), 
                  %als auch der fails (Lokalisierung nur in Mitte des 
                  %Raumes zuverlässig, Balldetection gibt 
                  %falsePositives wenn weißes Licht zuviele rote 
                  %Partikel enthält) 

\chapter{Zusammenfassung}
\label{cha:zusammenfassung}
Unsere Aufgabenstellung eine koordinierte Ballsuche 
zu implementieren, ist uns im Wesentlichen gelungen:\\
Ein zentraler Server nimmt die Bahnplanung vor und ist in der Lage die
Bahnplanung für bis zu drei Clients vorzunehmen. Diese Bahn wird von
der Clientsoftware ohne größere Probleme abgefahren. Mit der Hilfe der
uns zur Verfügung stehenden Ballerkennung und des
auf einen früheren Teamprojekt aufbauenden Sonar-Partikelfilters war
es möglich, den Client so zu
entwickeln, dass er nicht nur seine eigene Position im Raum, sondern
auch eventuell dort vorhandene Bälle bestimmen kann. \\\\
Zwar erwies sich der Partikelfilter in bestimmten Situationen als nur
bedingt zuverlässig, diese Problematik lässt sich aber durch ein
geschickte Wahl der Initialisierungsposition umgehen. Theoretisch wäre
es auch denkbar, die Roboter zusätzlich zum eingebauten Sonar mit
einen Laserscanner auszustatten, dieser sollte vom Ansatz her deutlich
zuverlässigere Ergebnisse liefern. Eine Implementierung dieses
Ansatzes lag allerdings nicht im Fokus unseres Projektes. \\\\
Grundsätzlich kann also das Projekt als Erfolg betrachtet werden.
%%% Local Variables:
%%% mode: latex
%%% TeX-master: "template"
%%% End:
 % Zusammenfassung halt :) 
%\newpage
%\setstretch{1}
 
% \begin{appendix} 
% dazu anhang mit installationsanleitung und ggf. weiteren Kram. 
%\section{Integration bestehender Projekte in ein neues Buildsystem für den
Client}
\label{cha:integr-best-proj}
\subsection{Warum ein neues Buildsystem?}
Die Ballerkennung und der Sonarpartikelfilter hängen von vielen
Altprojekten
ab. Das Beispiel für die Steuerung der Pioneer 3-DX hing zudem von der ARIA
Bibliothek\footnote{http://robots.mobilerobots.com/wiki/ARIA} ab. Bei den
Altprojekten fanden sich auch zum Teil fest kodierte Pfade. Da auch die
Konvertierung der Solutions in das VS 2010 Format meistens fehlschlug und
auch
ARIA nicht mit der mitgelieferten Solution unter VS 2010 gebaut werden
konnte,
haben wir uns für den Client für den Einsatz eines neuen Buildsystems
entschlossen. Aufgrund der Einfacheit der Beschreibung der Buildvorgänge
für
die einzelnen zu integrierenden Projekte und dem Hinzufügen neuer
Unterprojekte, sowie der Möglichkeit, von einem konkreten Buildsystem
abhängig
zu sein sowie existierender Erfahrungen mit dem System, fiel die Wahl auf
CMake\footnote{http://www.cmake.org/}. Bei CMake handelt es sich um ein
open-source Metabuildsystem. Es erzeugt Build-Vorschriften für andere
Buildsysteme (u.a. GNU make, nmake, msbuild). Somit wäre es bei Bedarf auch
Möglich gewesen, auf eine andere Visual Studio Version umzusteigen ohne
die Solutions zu konvertieren oder neu zu erstellen.

\subsection{Integration der Projekte}
Zuerst wurde eine passende Ordnerstrukur angelegt: include für
Header-Dateien,
src für die cpp-Dateien der einzelnen Projekt, bin für eventuell zum Linken
 benötigte
Kompillate, deren Quellen nicht in den Build-Prozess integrierbar waren und
build als Zielordner für erzeugte Buildfiles.

Als nächstes wurde für das Projekt im Ordner des Clients eine
CMakeLists.txt-Datei angelegt:

\begin{lstlisting}[language={},captionpos=b,caption={CMakeLists.txt für das Buildsystem des Clients}]
cmake_minimum_required (VERSION 2.6)
project (TP_WS_2010)

include_directories (include)
include_directories (include/ControlExample)
include_directories (include/erob/)
include_directories (include/Balldetection)
include_directories (include/p3dxSteuerung)
include_directories (include/StochasticLib)
include_directories (include/PioneerMRClient)
include_directories (include/Aria)
include_directories (include/OccupancyGridMap)
include_directories (include/irpVideo/DirectShow)
link_directories (${TP_WS_2010_BINARY_DIR})
link_directories (${TP_WS_2010_SOURCE_DIR}/lib/)
add_subdirectory (src)
\end{lstlisting}

Diese definiert den Namen des Projektes (wichtig
für Referenzen auf z.b. das Stammverzeichnis des Projektes oder das
Buildverzeichnis, da die entprechenden Variablennamen mit dem Projektnamen
+ \_
als Präfix versehen werden. Zudem wurden die include-Verzeichnisse und zum
Linken relevante Verzeichnisse festgelegt. Danach wurde das src-Verzeichnis
 als
Projektunterordner hinzugefügt. Dadurch werden die Anweisungen in der
CMakeLists.txt des Unterordners auch ausgeführt. Dies ermöglicht es, für
die
einzelnen Unterprojekte jeweils eine eigene kleine CMakeLists.txt zu
verwenden.

Die CMakeLists.txt bindet nun alle Sourceordner der Unterprojekte ein:

\begin{lstlisting}[language={},captionpos=b,caption={CMakeLists.txt für den Sourceordner}]
%\begin{lstlisting}
cmake_minimum_required (VERSION 2.6)

add_subdirectory (irpUtils)
add_subdirectory (irpMath)
add_subdirectory (irpImage)
add_subdirectory (irpVideo)
add_subdirectory (irpFeatureExtraction)
add_subdirectory (alglib)
add_subdirectory (DistanceMap)
add_subdirectory (irpCamera)
add_subdirectory (irpV3d)
add_subdirectory (StochasticLib)
add_subdirectory (Visualization)
add_subdirectory (CamCalib)
add_subdirectory (Balldetection)
add_subdirectory (mathtest)
add_subdirectory (SonarParticleFilter)
add_subdirectory (PioneerMRClient)
add_subdirectory (OccupancyGridMap)
add_subdirectory (Balldetection_Test)
add_subdirectory (Aria)
add_subdirectory (p3dxSteuerung)
add_subdirectory (p3dxSteuerung_threaded)
add_subdirectory (calibrate)
#add_subdirectory (ControlExample)
add_subdirectory (utils)
add_subdirectory (kinect)
add_subdirectory (networkTest)
\end{lstlisting}

Bei den Unterprojekten gibt es zwei Arten: Bibliotheken und Anwendungen.

Die CMakeList.txt einer Bibliothek sieht wie folgt aus:
\begin{lstlisting}[language={},captionpos=b,caption={CMakeLists.txt einer Bibliothek am
    Beispiel irpCamera}]
%\begin{lstlisting}
cmake_minimum_required (VERSION 2.6)

file(GLOB src "*.cpp")
file(GLOB includes "../../include/irpCamera/*.h")

add_library (irpCamera ${src} ${includes})
\end{lstlisting}

Die CMakeList.txt einer Anwendung sieht so aus:
\begin{lstlisting}[language={},captionpos=b,caption={CMakeLists.txt des Clients}]
cmake_minimum_required (VERSION 2.6)

file(GLOB src "*.cpp")
file(GLOB includes "../../include/p3dxSteuerung/*.h")
add_executable (p3dxSteuerung_threaded ${src} ${includes})
target_link_libraries (p3dxSteuerung_threaded irpUtils MINPACK fftw rfftw
DirectShow irpFeatureExtraction irpMath  alglib irpImage irpCamera irpVideo
 irpV3d glew CamCalib Visualization StochasticLib OccupancyGridMap
SonarParticleFilter ws2_32 winmm advapi32 Aria DistanceMap Balldetection)

\end{lstlisting}

Zuerst werden alle cpp-Dateien zur Variablen src hinzugfügt, danach die
Includes zur Variablen include. Je nachdem, ob es sich um eine Bibliothek
oder
Anwendung handelt, wird diese mit add\_library(name dateien) oder
add\_executable(name dateien) hinzugefügt. Die Include-Dateien müssen dabei
aber eigentlich nicht explizit hinzugefügt werden, dies geschieht nur,
damit
sie auch in einer VS Solution in der Dateiliste auftauchen. Bei Anwendungen
können nun mit target\_link\_libraries(exename libraries) noch die zu
linkenden
Bibliotheken angegeben werden.

Es gab ein paar Probleme bei der Integration: zum einen mussten zuerst die
Abhängigkeiten zwischen den Unterprojekten ermittelt werden, dies geschah
mit
Hilfe der originalen VS Solutions sowie durch Ausprobieren. Zum anderen gab
 es
manchmal Dateien, die zwar in den Quellen lagen, jedoch in nicht in den
Solutions auftauchten. Diese konnten den Buildprozess stören und mussten
daher
entfernt werden TODO: Beispiel raussuchen, wenn noch überhaupt möglich.
Außerdem waren an ein paar Stellen kleinere Änderungen am Quellcode nötig,
um
ihn mit MSVC 2010 oder der Struktur unseres include-Ordners kompatibel zu
machen.

\subsection{Hinzufügen eines neuen Unterprojektes}
\label{sec:hinz-eines-neuen-unterprojektes}
Das Hinzufügen eines neuen Unterprojektes ist relativ einfach.
Zuerst wird ein Unterverzeichnis für das Projekt im include-Ordner und
eines im
src-Ordner angelegt. Dann kopiert man die includes in den neuen Unterordner
 im
include-Ordner. Dabei ist darauf zu achten, dass die Pfade in
\#include-Direktiven relative zum include-Ordner sein sollten. Danach
kopiert
man die cpp-Dateien in den neuen Unterordner im src-Verzeichnis.

Danach erstellt man in diesem Ordner eine CMakeLists.txt ähnlich wie oben,
jenachdem ob ex sich um eine Anwendung oder Bibliothek handelt. Man fügt
zuerst
die cpp-Dateien und Includes hinzu, fügt dann mit add\_library oder
add\_executable Buildtargets hinzu und gibt eventuell noch zu linkende
Bibliotheken bei Anwendungen an mit target\_link\_libraries.

%%% Local Variables: 
%%% mode: latex
%%% TeX-master: "template"
%%% End: 
 
%\end{appendix} 
 
%literatur? Vielleicht die quelle des bahnplanungsalgorithmuses? 
%\begin{thebibliography}{99} 
%        \addcontentsline{toc}{chapter}{Bibliography} 
%        \bibitem[HORST06]{horst}\textsc{Horstmann, Werner}: 
%        \textsl{Das Fressverhalten der gemeinen Steinlaus bei Gegenlicht}. 
%        Springer, Berlin, March 2006. 
% 
% 
%\end{thebibliography} 
 
 
 
\end{document} 
 
 
%%% Local Variables:  
%%% mode: latex 
%%% TeX-master: t 
%%% End:  
