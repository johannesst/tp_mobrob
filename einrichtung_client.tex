
\subsection{Einrichtung des Clients}%Konfiguration und Nutzung des Clients}
\label{sec:einrichtung-client}
Zunächst müssen die Webcam des Roboters und dessen serielle
Schnittstelle  mit dem Laptop, auf dem die Clientsoftware läuft,
verbunden  werden. Anschließend wird der Client über  die Datei \verb|config.cfg| eingerichtet. Eine
Erklärung und ein Beispiel finden sich im Abschnitt
\ref{sec:funktionsweise} auf Seite \pageref{aufbau_config}. Die Datei
muss sich im gleichen Verzeichnis wie die ausführbare Datei
\verb|p3dxSteuerung_threaded.exe| befinden. Ausserdem benötigt der
Client ein Batchskript zum Starten, die
Daten der Kalibrierung und Karten für die Lokalisierung. Idealerweise
erzeugt man sich für die *exe-Datei und anderen benötigten Dateien ein
eigenes Verzeichnis, wo alle benötigten Dateien rein kopiert werden.  Hierzu
empfiehlt sich folgender Aufbau des Laufzeitverzeichnisses: 

