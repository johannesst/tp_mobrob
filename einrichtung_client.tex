
\subsection{Einrichtung des Clients}%Konfiguration und Nutzung des Clients}
\label{sec:einrichtung-client}
Zunächst müssen die Webcam des Roboters und dessen serielle
Schnittstelle  mit dem Laptop, auf dem die Clientsoftware läuft,
verbunden  werden. Anschließend wird der Client über  die Datei \verb|config.cfg| eingerichtet. Eine
Erklärung und ein Beispiel finden sich im Abschnitt
\ref{sec:funktionsweise} auf Seite \pageref{aufbau_config}. Die Datei
muss sich im gleichen Verzeichnis wie die ausführbare Datei
\verb|p3dxSteuerung_threaded.exe| befinden. Ausserdem benötigt der
Client ein Batchskript zum Starten, die
Daten der Kalibrierung und Karten für die Lokalisierung. Idealerweise
erzeugt man sich für die *exe-Datei und anderen benötigten Dateien ein
eigenes Verzeichnis, wo alle benötigten Dateien rein kopiert werden.  Hierzu
empfiehlt sich der in Abbildung \ref{aufbau_laufzeit} skizzierte
Aufbau des Laufzeitverzeichnisses. Die Bedeutung der Elemente der
Verzeichnisstruktur ist der Tabelle \ref{bedeutung_namen} zu entnehmen.
\begin{nofloat}{figure}%{l}{1\textwidth}
\label{aufbau_laufzeit}
\centering
\begin{tikzpicture}
%\tikz 
[font=\footnotesize,
       grow=right, level 1/.style={sibling distance=5em}
                   level 2/.style={sibling distance=6em}, level distance=5cm]
  \node {p3dxSteuerung} % root
     child { node {config}}
     child {node {client}
       child {node {Aria.dll}}
       child {node {calib.exe}}
       child {node {Calibrate.cmd}}
       child {node {rayCasting}
         child{node {Elevator2cm.bmp}}
         child{node {Elevator.BMP}}
         child{node {OccuMap.bmp}}
       }
       child {node {config.cfg}}
       child {node{balldetection\_test.exe}}
       child {node {p3dxSteuerung\_threaded.exe}}
      % child {node {CamNo}}
       child {node {Start.bat}}
       child {node {wskKarte}
         child{node{Box.BMP}}
         child{node{Box.txt}}
         child{node {Elevator2.txt}}
         child{node {Elevator.BMP}}
         child{node {Elevator.txt}}
         }
       child {node {wtee.exe}}
   };
\end{tikzpicture}
\caption{Empfohlener Aufbau des Laufzeitverzeichnisses}  
\end{nofloat}
\begin{nofloat}{table}{
    %\begin{table}
      \centering
      \begin{tabular}{|c|p{0.8\linewidth}|}
        \hline 
        Name & Bedeutung \\ \hline
        config & Vereichnis mit den *xml-Kalibrierungsdateien\\ \hline
        client & Verzeichnis mit den Programmdateien des Clients\\ \hline
        Aria.dll & DLLs, die vom Client zur Ansteuerung
        des Roboters
        \\ \hline
        %Datei der libAria, wird zur Ansteuerung des
        %Roboters benötigt\\ \hline
        calib.exe & Hilfstool zur Kalibrierung der Ballerkennung
        \footnote{Basierend auf der balldetection\_test.exe von Tobias Breuer}\\ \hline
        Calibrate.cmd &  Wrapper-Batchskript für die Kalibrierung \\ \hline
        rayCasting,  wskKarte & Verzeichnisse mit Raumkarten im
        BMP-Format\\ \hline
        config.cfg & Konfigurationsdatei für den Client\\ \hline
        balldetection\_test.exe & Miniprogramm zum Testen der
        Ballerkennung \footnote{Geschrieben von Tobias Breuer}\\ \hline
        wtree.exe & Windows-Tool, um Bildschirmausgaben von
        DOS-Programmen zu loggen und trotzdem am Bildschirm
        auszugeben.\footnote{OpenSource von Ryan Buhl unter der
          Mozilla Public License}\\ \hline
        Start.bat & Batch-Skript, um den Client auszuführen und
        gleichzeitig ein Logging mittels wtree.exe zu ermöglichen.
        %und gleichzeitigen
        %Ausgaben voder Ausgabe des Clients
        %\footnote{Dieses T
       \\ \hline
      \end{tabular}
      \caption{Die Dateien des Laufzeitverzeichnisses und ihre Bedeutung}
      \label{bedeutung_namen}
   % \end{table}
}
\end{nofloat}
%%% Local Variables: 
%%% mode: latex
%%% TeX-master: "template"
%%% End: 
