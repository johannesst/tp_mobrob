

\section{Erstellung von Client und Server mit Visual Studio  2010}
Da der Client mit Hilfe des im vorherigenen Abschnitts auf CMake
basierenden Buildystems erzeugt wird, sind hier mehr Schritte als beim
Server notwendig:
\begin{enumerate}
	\item CMake für Windows (Version min. 2.6) herunterladen und installieren
  \footnote{http://www.cmake.org/}
	\item CMake GUI starten
	\item Unter "Where is the source code" den Pfad des Client-Verzeichnisses
 eintragen und unter "Where to build the binaries" den Pfad zu dem
Verzeichnis, wo die Solution für den Client erstellt werden soll
	\item Auf "Configure" klicken, ,,Visual Studio 2010'' und ,,Use default
native compilers'' auswählen, dann auf Finish klicken. Auf das Ende des
Einrichtens warten.
	\item Auf "Generate" klicken.
	\item Im ausgewählten Zielverzeichnis befindet sich nun eine
          Visual Studio 2010
Solution (TP\_WS\_2010.sln). 
\item Außerdem befindet sich im Unterordner ,,PioneerMRServer'' im
Projektbaum eine Solution für den
Server (PioneerMRServer.sln). Beide Solutions können ganz normal mit 
Visual Studio 2010 geöffnet und gebaut werden (F7 oder STRG-Alt-F7).
\end{enumerate}

%%% Local Variables: 
%%% mode: latex
%%% TeX-master: "template"
%%% End: 
